%!TEX program = lualatex
%
% This file and all related are subject to copyright:
%
%   Copyright 2016 Timothy A. V. Teatro
%
%
\documentclass[11pt, letterpaper, oneside, openany, article]{memoir}
\usepackage{xcolor}%!TEX root = Proposal-PhD.tex
%


% GitHub-ish 2016
% Tim Teatro
%  51  51  51  #333333
% 150 152 150  #969896
% 167  29  93  #a71d5d
%  24  54 145  #183691
% 121  93 163  #795da3
%   0 134 179  #0086b3
% 237 106  67  #ed6a43


\definecolor{git-dkgray}{HTML}{333333}
\definecolor{git-ltgray}{HTML}{969896}
\definecolor{git-red}{HTML}{a71d5d}
\definecolor{git-blue}{HTML}{183691}
\definecolor{git-purple}{HTML}{795da3}
\definecolor{git-turquoise}{HTML}{0086b3}
\definecolor{git-orange}{HTML}{ed6a43}

\definecolor{gitish-turquoise}{HTML}{004c80}
\definecolor{gitish-red}{HTML}{80004c}

\usepackage{tikz}
\usetikzlibrary{calc}
\usetikzlibrary{mindmap}
\PassOptionsToPackage{hyphens}{url}
\usepackage[pdftex,
  plainpages=false,
  pdfpagelabels,
  bookmarksnumbered,
  bookmarksopen,
  colorlinks, % Removes boxes and colours links, but shows in print off.
  linkcolor={gitish-red},
  citecolor={gitish-red},
  urlcolor ={gitish-red},
  ]{hyperref}
\usepackage{memhfixc}
\usepackage[
  backend=biber,
  style=ieee,
  datamodel=standard,
  ]{biblatex}
\usepackage{microtype}
\usepackage{ragged2e}
\usepackage[xindy]{imakeidx}
\usepackage[shortlabels]{enumitem}
\usepackage[english]{datetime2}


\makeatletter



%%% Font selection and related definitions
%====================================================================


\usepackage{amsmath, mathtools}
\usepackage{pgfornament}
\usepackage{fontspec}
\defaultfontfeatures{RawFeature=+calt}
\usepackage{lualatex-math}
\usepackage[math-style=ISO, partial=upright]{unicode-math}

\setmainfont[
  Path=/home/timtro/.fonts/type1/Libertinus/,
  ItalicFont=libertinusserif-italic.otf,
  BoldFont=libertinusserif-bold.otf,
  BoldItalicFont=libertinusserif-bolditalic.otf,
    ]{libertinusserif-regular.otf}
\setsansfont[
  Path=/home/timtro/.fonts/type1/Libertinus/,
  ItalicFont=libertinussans-italic.otf,
  BoldFont=libertinussans-bold.otf,
    ]{libertinussans-regular.otf}
\setmonofont[Scale=MatchLowercase]{Hack}

\newfontfamily{\displayfamily}[Path=/home/timtro/.fonts/type1/Libertinus/]{libertinusserifdisplay-regular.otf}
\newfontfamily{\semibfseries}[Path=/home/timtro/.fonts/type1/Libertinus/]{libertinusserif-semibold.otf}
\newfontfamily{\semibfit}[Path=/home/timtro/.fonts/type1/Libertinus/]{libertinusserif-semibolditalic.otf}

\setmathfont[Path=/home/timtro/.fonts/type1/Libertinus/, AutoFakeBold]{libertinusmath-regular.otf}
% \setmathfont[range={\lbrace, \rbrace}]{Latin Modern Math}
% \setmathfont[range={\vert, \Vert, \Vvert}]{Latin Modern Math}

\captiontitlefont{\footnotesize\sffamily}
\captionnamefont{\footnotesize\sffamily}

\usepackage{lettrine}
\newfontfamily{\infamily}{Roboto Condensed}[Scale=MatchLowercase]
\renewcommand*{\DefaultLoversize}{0.1}
\renewcommand{\LettrineFontHook}{\infamily}
\renewcommand{\LettrineTextFont}{\sffamily\MakeTextUppercase}

\usepackage{marginnote}
\renewcommand{\marginfont}{\footnotesize}

\usepackage{siunitx}

\newcommand{\asterismFrame}[1]{\par\bigskip\noindent\hfill #1 \hfill\null\par\bigskip\@afterindentfalse\@afterheading}
\newcommand{\linearAsterism}{\asterismFrame{$*\quad*\quad*$}}
\newcommand{\bouquetAsterism}{\asterismFrame{%
    \begin{tikzpicture}
      \node[anchor=centre, rotate=120]{\pgfornament[width=.5cm]{9}}
    \end{tikzpicture}
  }
}
\newcommand{\fleurAsterism}{\asterismFrame{\adfflowerright}}
\newcommand{\sAsterism}{\asterismFrame{\adfopenflourishleft}}
\newcommand{\ssAsterism}{\asterismFrame{\adfdoubleflourishleft}}
\newcommand{\sssAsterism}{\asterismFrame{\adftripleflourishleft}}
\newcommand{\sickleAsterism}{\asterismFrame{\adfsickleflourishleft}}



%% Code between BEGIN and END tTechnicalNoteCopy was taken from my
%% tTechnicalNote class. It exists here with modifications for this document.

%%%
%%%  %% BEGIN tTechnicalNoteCopy
%%%

%%% Page layout
%====================================================================


\settrims{0pt}{0pt}
\settypeblocksize{49\onelineskip}{130mm}{*}
\setlrmargins{10.45mm}{*}{1} % 15.45 = (215.9mm - 5mm - 60mm - 130mm)/2
\setulmargins{*}{*}{1}
\setmarginnotes{5mm}{60mm}{5pt}
\setheadfoot{\baselineskip}{9mm}
\setheaderspaces{*}{9mm}{*}
\checkandfixthelayout



%%% Numbered margin notes.
%====================================================================


\newcounter{sideNoteNum}[chapter]
\NewDocumentCommand{\SideNote}{mO{0pt}}{%
        \refstepcounter{sideNoteNum}%
        \mbox{\textsuperscript{\arabic{sideNoteNum}}}%
        \marginnote{\indent\mbox{\normalsize\textsuperscript{\arabic{sideNoteNum}}}\hspace{0pt}~\footnotesize #1}[#2]%
}
\renewcommand*{\raggedleftmarginnote}{\RaggedRight}
\renewcommand*{\raggedrightmarginnote}{\RaggedRight}



%%% Theorems, proofs, defs, etc.
%====================================================================


\usepackage{amsthm}
\usepackage[framemethod=tikz]{mdframed}

\newtheoremstyle{plain}
  {\topsep}    % ABOVESPACE
  {\topsep}    % BELOWSPACE
  {\itshape}  % BODYFONT
  {0pt}        % INDENT (empty value is the same as 0pt)
  {\bfseries}  % HEADFONT
  {.}          % HEADPUNCT
  {5pt plus 1pt minus 1pt} % HEADSPACE
  {}          % CUSTOM-HEAD-SPEC

\newtheoremstyle{definition}
  {\topsep}    % ABOVESPACE
  {\topsep}    % BELOWSPACE
  {\normalfont}  % BODYFONT
  {0pt}        % INDENT (empty value is the same as 0pt)
  {\bfseries}  % HEADFONT
  {.}          % HEADPUNCT
  {5pt plus 1pt minus 1pt} % HEADSPACE
  {}          % CUSTOM-HEAD-SPEC

\newtheoremstyle{remark}
  {.5\topsep}    % ABOVESPACE
  {.5\topsep}    % BELOWSPACE
  {\normalfont}  % BODYFONT
  {0pt}        % INDENT (empty value is the same as 0pt)
  {\itshape}  % HEADFONT
  {.}          % HEADPUNCT
  {5pt plus 1pt minus 1pt} % HEADSPACE
  {}          % CUSTOM-HEAD-SPEC

\theoremstyle{plain}

\newtheorem*{identity*}{Identity}
\surroundwithmdframed[
  hidealllines=true,
  innerleftmargin=1em,
  innerrightmargin=1em,
  innertopmargin=0pt,
  ]{identity*}

\newmdtheoremenv[
  hidealllines=true,
  innerleftmargin=1em,
  innerrightmargin=1em,
  innertopmargin=0pt,
  ]{identity}{Identity}[chapter]

\newmdtheoremenv[
  hidealllines=true,
  leftline=true,
  innerleftmargin=10pt,
  innerrightmargin=10pt,
  innertopmargin=0pt,
  ]{definition}{Definition}

\theoremstyle{definition}
\newmdtheoremenv[
  hidealllines=true,
  innerleftmargin=1em,
  innerrightmargin=1em,
  subtitlefont={\itshape},
  ]{displayAlgorithm}{Algorithm}[chapter]

\theoremstyle{remark}
\newtheorem*{remark*}{Remark}
\surroundwithmdframed[
  hidealllines=true,
  leftline=true,
  innerleftmargin=10pt,
  innerrightmargin=10pt,
  innertopmargin=0pt,
  ]{remark*}

\NewDocumentCommand{\MarginDefinition}{mO{0pt}}{%
   \refstepcounter{definition}%
   \@afterindentfalse\@afterheading%
   \marginnote{\rule{\marginparwidth}{2pt}\smallbreak\noindent\footnotesize\textbf{Definition \thedefinition.}\hspace{0.5em plus 0.25em minus .1em} #1\smallbreak%
   \noindent\rule{\marginparwidth}{.5pt}}[#2]%
}



%%% Chapter Style
%====================================================================


\chapterstyle{section}
\renewcommand{\chaptitlefont}{\raggedright\color{black!85}\Huge\bfseries\sffamily}
\setsecheadstyle{\Large\sffamily}
\setsubsecheadstyle{\large\itshape}



%%% Page Style
%====================================================================


\makepagestyle{technote}
\setlength{\headwidth}{\textwidth}
  \addtolength{\headwidth}{\marginparsep}
  \addtolength{\headwidth}{\marginparwidth}
\makerunningwidth{technote}{\headwidth}
\makeheadrule{technote}{\headwidth}{\normalrulethickness}
\makefootrule{technote}{\headwidth}{\normalrulethickness}{5pt}
\makeheadposition{technote}{flushright}{flushleft}{flushright}{flushleft}
\makepsmarks{technote}{%
  \def\chaptermark##1{\markboth{##1}{##1}}    % left mark & right marks
  \def\sectionmark##1{\markright{%
    \ifnum \c@secnumdepth>\z@
      \thesection. \ %
    \fi
    ##1}}
  \def\tocmark{\markboth{\contentsname}{\contentsname}}
  \def\lofmark{\markboth{\listfigurename}{\listfigurename}}
  \def\lotmark{\markboth{\listtablename}{\listtablename}}
  \def\bibmark{\markboth{\bibname}{\bibname}}
  \def\indexmark{\markboth{\indexname}{\indexname}}}
\makepsmarks{technote}{%
  \nouppercaseheads
  \createmark{chapter}{both}{nonumber}{}{}
  \createmark{section}{right}{shownumber}{}{. \space}
  \createplainmark{toc}{both}{\contentsname}
  \createplainmark{lof}{both}{\listfigurename}
  \createplainmark{lot}{both}{\listtablename}
  \createplainmark{bib}{both}{\bibname}
  \createplainmark{index}{both}{\indexname}
  \createplainmark{glossary}{both}{\glossaryname}}
\makeevenhead{technote}{\normalfont\leftmark}{}%
                        {\normalfont\thetitle}
\makeoddhead{technote}{\normalfont\leftmark}{}%
                       {\normalfont\thepage}
\makeevenfoot{technote}{\scriptsize ©\the\year\ \@author\ — All Rights Reserved}{}{}
\makeoddfoot{technote}{\scriptsize ©\the\year\ \@author\ — All Rights Reserved}{}{}

\aliaspagestyle{chapter}{technote}
\setlength{\headwidth}{\textwidth}
\addtolength{\headwidth}{\marginparsep}
\addtolength{\headwidth}{\marginparwidth}
\pagestyle{technote}



%%% Make Title
%====================================================================


\newcommand{\theNoteType}{Technical Note}
\NewDocumentCommand{\NoteType}{m}{\renewcommand{\theNoteType}{#1}}

\aliaspagestyle{title}{technote}

\renewcommand{\@maketitle}{%
    \let\footnote\@mem@titlefootkill
    \ifdim\pagetotal>\z@
      \newpage
    \fi
    \null
    \noindent\begin{tabular}{ll}
      \textbf{Author:} & \@author\\
      \textbf{Date:}   & \@date\\
    \end{tabular}
    \vskip 1cm
    \begin{center}
      {\bfseries\theNoteType}\\
      \Large\scshape\@title
    \end{center}
    \vskip 5mm
  }

\makeatother



%%% Figures
%====================================================================


\usepackage[wide, outercaption]{sidecap}
  \renewcommand{\sidecaptionsep}{5mm}



%%% Text macros
%====================================================================


\NewDocumentCommand{\anno}{m}{\oldstylenums{#1}}
\NewDocumentCommand{\decade}{m}{\anno{#1}s}


%%%
%%%  %% END tTechnicalNoteCopy
%%%

%%% Figures, units, captions, listings
%====================================================================


\usepackage{listings} % for code listings
\lstloadlanguages{[11]C++}
\lstdefinestyle{C++GIT}{ %
  language=[11]C++,
  xleftmargin=\parindent,
  basewidth=0.57em,
  breakatwhitespace=true,
  breaklines=true,
  keepspaces=true,
  showstringspaces=true,
  showtabs=false,
  tabsize=2,
  basicstyle=\small\ttfamily,
  directivestyle={\color{gitish-red}},
  keywordstyle=\bfseries\color{gitish-red},
  commentstyle=\itshape\color{git-ltgray},
  identifierstyle=\color{git-dkgray},
  stringstyle=\color{git-orange},
  numberstyle=\color{git-orange},
  emphstyle=\color{gitish-turquoise},
  literate=%
    {0}{{{\color{git-orange}0}}}1
    {1}{{{\color{git-orange}1}}}1
    {2}{{{\color{git-orange}2}}}1
    {3}{{{\color{git-orange}3}}}1
    {4}{{{\color{git-orange}4}}}1
    {5}{{{\color{git-orange}5}}}1
    {6}{{{\color{git-orange}6}}}1
    {7}{{{\color{git-orange}7}}}1
    {8}{{{\color{git-orange}8}}}1
    {9}{{{\color{git-orange}9}}}1
}
\lstset{
  language=[11]C++,
  style=C++GIT,
  morekeywords={override},
  moreemph={std, void, string, int, float, double, vector}
}
\lstMakeShortInline{|}

\usepackage{lstlinebgrd}

\usepackage{multirow}
\usepackage{tikz-uml} % Must be included after \pdfminorversion is set.
\tikzumlset{font=\scriptsize\ttfamily}
\tikzumlset{fill class = white, fill note = white}
\usepackage{algorithmicx, algpseudocode}
\usepackage{import} % for Inkscape figs.
\usepackage{transparent}



%%% Bibliography stuff
%====================================================================


\addbibresource{library.bib}
\addbibresource{standards.bib}
\addbibresource{software.bib}



%%% Epigraph stuff...
%====================================================================


\usepackage{xparse}
\usepackage{adforn}
\NewDocumentCommand\EpiName{m}{\adforn{64}~\textsc{#1}\\}
\NewDocumentCommand\EpiSource{m}{\emph{#1}\\}
\NewDocumentCommand\EpiBio{m}{\normalfont #1\\}
\setlength\epigraphwidth{.75\textwidth}
\setlength\epigraphrule{0pt}

\newcommand{\epiAsterismFrame}[1]{\par\smallskip\noindent\hfill #1 \hfill\null\par\smallskip\@afterindentfalse\@afterheading}
\newcommand{\epiAsterism}{\epiAsterismFrame{\adfopenflourishleft}}



%%% Drafting aids:
%====================================================================


% \usepackage{everypage}
% \AddEverypageHook{
%   \begin{tikzpicture}[remember picture,overlay]
%     \node [yshift=.75cm] at (current page.south) {\sffamily\color{black!40}\large Draft of \Today};
%   \end{tikzpicture}
% }
% \usepackage[colorinlistoftodos]{todonotes}
% \presetkeys{todonotes}{color={white}, linecolor={git-turquoise}, bordercolor={gitish-turquoise}}{}
% \listfiles


\makeatother



%====================================================================
%====================================================================
%====================================================================
\makeindex
%!TEX root = Proposal-PhD.tex
%

\newcommand{\NewMathSymbol}[2]{\newcommand{#1}{\ensuremath{#2}}}


% Names

\newcommand{\virtualmeR}{virtualME\textsuperscript{®}}


% Operators
\NewMathSymbol{\dif}{\mathup{d}}
\newcommand{\tr}{\ensuremath{^\mathsf{T}}}
\newcommand{\bi}[1]{\ensuremath{\mathbfit{#1}}}
\newcommand{\pdop}[1]{\partial_{#1}}
\NewDocumentCommand{\ipdiff}{mm}{\ensuremath{\partial #1/\partial #2}}
\NewDocumentCommand{\pdiff}{omm}{\ensuremath{%
  \IfNoValueTF{#1}{%
    \frac{\partial{#2}}{\partial{#3}}%
  }{%
    \frac{\partial^{#1}{#2}}{\partial{#3}^{#1}}%
  }%
}}
\NewDocumentCommand{\diff}{omm}{\ensuremath{%
  \IfNoValueTF{#1}{%
    \frac{\dif{#2}}{\dif{#3}}%
  }{%
    \frac{\dif^{#1}{#2}}{\dif{#3}^{#1}}%
  }%
}}
\NewDocumentCommand{\popdiff}{omm}{\ensuremath{%
  \IfNoValueTF{#1}{%
    \pdop{#3} #2%
  }{%
    \pdop{#3}^#1 #2%
  }%
}}
\NewDocumentCommand{\intco}{mm}{\ensuremath{[#1,\,#2)}}
\NewDocumentCommand{\intcc}{mm}{\ensuremath{[#1,\,#2]}}
\DeclareMathOperator*{\argmin}{arg\,min}
\newcommand{\diag}{\mathop{\mathrm{diag}}}


\let\oldnorm\norm   % <-- Store original \norm as \oldnorm
\let\norm\undefined % <-- "Undefine" \norm
% \DeclarePairedDelimiter\norm{\lVert}{\rVert}
\NewDocumentCommand{\norm}{m}{\lVert #1\rVert}

\renewcommand{\implies}{\Rightarrow}

% Mathematical Constants
\newcommand{\ball}[1]{\ensuremath{\mathcal{B}}_{#1}}
\NewMathSymbol{\iu}{\mathrm{i}}%\mkern1mu}
\NewMathSymbol{\piup}{\symup{\pi}}
\NewMathSymbol{\expe}{\mathup{e}}
\newcommand{\eye}[1]{\ensuremath{\mathbfup{I}_{#1}}}

% General symbols
\NewMathSymbol{\bia}{\mathbfit{a}}
\NewMathSymbol{\biA}{\mathbfit{A}}
\NewMathSymbol{\biB}{\mathbfit{B}}
\NewMathSymbol{\bib}{\mathbfit{b}}
\NewMathSymbol{\bif}{\mathbfit{f}}
\NewMathSymbol{\bigee}{\mathbfit{g}}
\NewMathSymbol{\bih}{\mathbfit{h}}
\NewMathSymbol{\bip}{\mathbfit{p}}
\NewMathSymbol{\biP}{\mathbfit{P}}
\NewMathSymbol{\bis}{\mathbfit{s}}
\NewMathSymbol{\biu}{\mathbfit{u}}
\NewMathSymbol{\biU}{\mathbfit{U}}
\NewMathSymbol{\biv}{\mathbfit{v}}
\NewMathSymbol{\biq}{\mathbfit{q}}
\NewMathSymbol{\bir}{\mathbfit{r}}
\NewMathSymbol{\biy}{\mathbfit{y}}
\NewMathSymbol{\bix}{\mathbfit{x}}
\NewMathSymbol{\biX}{\mathbfit{X}}
\NewMathSymbol{\biY}{\mathbfit{Y}}
\NewMathSymbol{\biz}{\mathbfit{z}}
\NewMathSymbol{\biOmega}{\symbfit{\Omega}}
\NewMathSymbol{\dirac}{\symup{\delta}}
\NewMathSymbol{\ival}{t_\symup{\Delta}}
% \NewMathSymbol{\ival}{\symit{\Delta}}
\NewMathSymbol{\kron}{\symup{\delta}}
\NewMathSymbol{\ldif}{\symup{\delta}}

% Sets
\NewMathSymbol{\reals}{\mathbfup{R}}
\NewMathSymbol{\posReals}{\mathbfup{R}_+}
\NewMathSymbol{\nonNegReals}{\mathbfup{R}_{\ge0}}
\NewMathSymbol{\ints}{\mathbfup{Z}}
\NewMathSymbol{\posInts}{\mathbfup{Z}_+}
\NewMathSymbol{\nonNegInts}{\mathbfup{Z}_{\ge0}}
\NewDocumentCommand{\lebesgueIntegrable}{O{\infty}mO{}}{\ensuremath{\mathbfup{L}^{#3}_{#1,\,\text{loc}}#2}}
\NewDocumentCommand{\bigLebesgueIntegrable}{O{\infty}mO{}}{\ensuremath{\mathbfup{L}^{#3}_{#1,\,\text{loc}}\bigl(#2\bigr)}}
\NewDocumentCommand{\continuouslyDifferentiable}{O{}}{\ensuremath{\mathbfup{C}^{#1}}}

% For giving (optional) alignent options to AMS matrices.
\makeatletter
\renewcommand*\env@matrix[1][*\c@MaxMatrixCols c]{%
  \hskip -\arraycolsep
  \let\@ifnextchar\new@ifnextchar
  \array{#1}}
\makeatother

% NMPC specific symbols
\NewMathSymbol{\q}{\mathbfit{x}}
\NewMathSymbol{\oq}{\q_{\star}}
\NewMathSymbol{\qcu}{\q_\cu}
\NewMathSymbol{\qerr}{\tilde\q}
\NewMathSymbol{\stateSpace}{X}
\NewMathSymbol{\p}{\symbfit{\lambda}}
\NewMathSymbol{\op}{\p_{\star}}
\NewMathSymbol{\cu}{\mathbfit{u}}
\NewMathSymbol{\ocu}{\mathbfit{u}_{\star}}
\NewMathSymbol{\ctrlSpace}{U}
\NewMathSymbol{\ou}{\cu_{\star}}
\NewMathSymbol{\fc}{\mathbfit{f}_\mathup{c}}
\NewMathSymbol{\fd}{\mathbfit{f}_{+}}
\NewMathSymbol{\fbk}{\symbfit{\mu}}
\NewMathSymbol{\terminalSet}{\Omega}
\newcommand{\rnc}{\ensuremath{_\text{ref}}}
\NewMathSymbol{\Jaug}{J_{\symup{\Lambda},N}}

\NewMathSymbol{\ham}{H}
\NewMathSymbol{\delt}{\tilde{\q}}
\NewMathSymbol{\mQ}{\mathbfit{Q}}
\NewMathSymbol{\mS}{\mathbfit{S}}
\NewMathSymbol{\mR}{\mathbfit{R}}
\NewMathSymbol{\mI}{\mathbf{I}}

\NewMathSymbol{\xrod}{z}
\NewMathSymbol{\xn}{n_\text{N}}
\NewMathSymbol{\xdn}{n_\text{C}}
\NewMathSymbol{\xtf}{\tau_\text{f}}
\NewMathSymbol{\xtm}{\tau_\text{m}}
\NewMathSymbol{\xI}{\nu_\text{I}}
\NewMathSymbol{\xX}{\nu_\text{X}}

\NewMathSymbol{\alphaf}{\alpha_\text{f}}
\NewMathSymbol{\alpham}{\alpha_\text{m}}
\NewMathSymbol{\gammaI}{\gamma_\text{I}}
\NewMathSymbol{\gammaX}{\gamma_\text{Xi}}
\NewMathSymbol{\sigmaX}{\sigma_\text{X}}
\NewMathSymbol{\lambdaI}{\lambda_\text{I}}
\NewMathSymbol{\lambdaX}{\lambda_\text{Xi}}
\NewMathSymbol{\lambdaC}{\lambda_\text{C}}
\NewMathSymbol{\blk}{\mathbf{B}}

\newcommand{\nom}[1]{\ensuremath{#1^\bullet}}
\newcommand{\mc}{\multicolumn{1}{c}}


\usepackage[tooltip={true}]{acro}
\input{ece-acronyms.tex}

\author{Timothy A.\ V.\ Teatro}
\date{\Today}
\NoteType{PhD Proposal}

\renewcommand{\theequation}{\oldstylenums{\arabic{equation}}}
\renewcommand{\theequation}{\oldstylenums{\arabic{chapter}.\arabic{equation}}}

\begin{document}%\showindexmarks\indexmarksytle{\tiny\sffamily}

  %!TEX root = Proposal-PhD.tex
% \title{Exploration and Advancement of Techniques for Predictive Control Systems Software Design and Construction}
\begin{titlingpage}
	\calccentering{\unitlength}
	\begin{adjustwidth*}{\unitlength}{-\unitlength}
		\null
		\vfill
		\begin{center}
			\Large
			{%
      EXPLORATION\\ \textit{and} ADVANCEMENT\\
      \textit{of} Techniques\\
      \textit{for}\\
      PREDICTIVE CONTROL SYSTEMS\\
      Software Design and Construction
      }\\[6ex]
			{\large\itshape A Thesis Proposal}\\[4ex]
			{\normalsize\textit{by} \uppercase{ Timothy A.V. Teatro}}\\[1ex]
			\vfill
			\includegraphics[width=1.0in]{/home/timtro/Documents/logos/UOIT_black_noTM}
		\end{center}
		\vfill
	\end{adjustwidth*}
	\makebox[0pt][c]{%
		\begin{tikzpicture}[remember picture, overlay]
			\draw ($(current page.south west)+(1.5in,1.00in)$)
				rectangle ($(current page.north east)-(1.5in,1.00in)$);
		\end{tikzpicture}
	}
\end{titlingpage}

  %!TEX root = Proposal-PhD.tex
%
%
% \chapter*{Abstract}

  \clearpage
  {
    \hypersetup{hidelinks}
    \setcounter{tocdepth}{2}
    \tableofcontents
    \clearpage
  }

  %!TEX root = Proposal-PhD.tex
%
\chapter{Introduction}%
\label{chap:introduction}

  %!TEX root = Proposal-PhD.tex
%
\chapter{Background}%
\label{chap:background}

In this section, the aim is to provide a sufficient set of descriptions to form
a reasonable mental picture of the embodiment of the proposed research. This
section opens with a literature review encompassing methods for software
description of architecture and pattern based design which are necessary
background for~\ref{contrib:design}. The contribution \ref{contrib:capital} is
motivated partially by the recent history of the C++ language, which is reviewed
next, followed by functional programming and \ac{frp}. The remainder of the
literature review covers various aspects of optimal control, which underlies all
of the contributions, but is particularly expressed in~\ref{contrib:theory}.

The next section, \Sref{sec:prior-art}, describes work that potentially overlaps
with the proposed contributions. The next section, \Sref{sec:literature-gaps}, I
specifically outline the gaps in literature and prior-art toward which I direct
my efforts—localising my research goals and contributions within the field of
engineering.

In the remaining sections, I provide a serviceable theoretical background in
optimal control and \ac{nmpc} which supports a detailed discussion of the
proposal for~\ref{contrib:theory}.



\section{Literature Review}%
\label{sec:lit-review}


\subsection{Software Architectural Description}


\textsc{The software design} (as opposed to the code) will be documented in the
form of
%
\begin{enumerate}
  %
  \item patterns,
  \item contracts, and
  \item idioms.
  %
\end{enumerate}
%
This will include, but are not limited to descriptions compliant with
\acf{ieee1016} and \acf{ieee420101}.

\MarginDefinition{A \textbf{\textit{precondition}}\index{Precondition
(software)} is a condition on the parameters of a method, or data within the
scope of the method that is expected to be true for the method to behave
properly. They are obligations of the user of the method.
\textbf{\textit{Postconditions}}\index{Postcondition (software)} are a similar
sort of expectation that must be true \emph{after} the method runs.
\textbf{\textit{Invariants}}\index{Invariant (software)} are conditions on
parameters and data that are true before \emph{and} after the method runs.}
%
Contracts are a communication tool. They are a precise and complete
specification of behaviour visible to the user. They include a description of
preconditions, postconditions and invariants. It is a specification of the
behaviour of a routine without defining implementations.

When software code is constrained by contracts, bugs are easier to define since
a bug will be a departure from contractually defined behaviour. Well defined
contracts make unit testing more straightforward (since it makes clear what to
unit test). Clearly defined contractual behaviour combined with the lawful
constraints of functional programming precipitates the opportunity for
mathematical analysis and provable correctness.

When you were learning to read, you started by learning the letters of the
language and the sounds they encode. Next, you begin sounding out words through
inspection of the letters. When you became good at reading, you no longer had to
concentrate on individual letters. Instead, you grouped them visually into whole
words. Finally, in high-school you were taught the anatomy of various forms of
literature to tell stories.

Software patterns are to software what archetypes are to literature. The
programming language provides the dictionary of words, from which your code
forms the sentences. But, the story is made of archetypes: the evil
super-villain, the hero's small home town, the lone cowboy. The author is not
constrained by archetypes, and is free to adapt and reinterpret them. But
archetypes provide useful abstractions so that an expert reader is implicitly
granted insight. Software design patterns similarly facilitate unwritten
communication between program authors and the reader, and that is what design
patterns are about.

Patterns are small design units, of the sort that could be represented with
\ac{uml} diagrams. They define a solution to common design problems. So, if you
see one used, you also know something about the problem the programmer was
trying to solve. The intent is communicated through the application of the
pattern. For example, if I use the \emph{abstract factory} pattern, (one of the
original \ac{gof} patterns~\cite{GOF}), and I write a class called
\verb|WidgetFactory|, the nomenclature makes clear to readers of my code what is
and how they should use the \verb|WidgetFactory|. It is self-documentation for
appropriately literate readers.

Because software patterns are a communication tool based on common solutions to
common problems, they are useless as a personal fabrication. They must be
identified in the structure of production software. A significant phase of
research will require me to survey existing code, looking for these types of
patterns. I will document, and make rigorous definitions for the patterns and
their use in control systems programming.



\subsection{Functional Programming}



Probably the most common and venerable introduction to
functional\index{Functional programming} programming is John Hughes classic
paper \citetitle{Hughes1989}~\cite{Hughes1989}. A bit of history is to be found
in David Turner's \citetitle{Turner2013}~\cite{Turner2013}.

Systematic approaches to explaining functional programming often begin the
chronology in the \decade{1930} with Alonzo Church and the
$\symup{\lambda}$-calculus~\cite{Church1936, Church1941}. The
$\symup{\lambda}$-calculus is a mathematical theory to generalise untyped
functions and make rigorous the rules for composition and substitution. This
naturally requires a way of describing functions anonymously, giving rise to the
\emph{lambda} binding notation. For example,  $\symup{\lambda}x.x^2$, binds free
occurrences of $x$ to $x^2$. This is the origin of the convention to name
anonymous functions \emph{lambdas} in programming.

Church attempted to demonstrate that functions could be used to encode numerals,
and could therefore be a formal basis for all mathematics.

LISP and Algol 60 are often cited as the first languages to implement concepts
from the $\symup{\lambda}$-calculus. But in \anno{1966}, Peter Landin published
a landmark paper called \citetitle{landin1966next}~\cite{landin1966next} in
which he gives semantics for ISWIM, a language based directly on Church's
$\symup{\lambda}$-calculus.

In \anno{1977}, John Backus's in his Turing award speech~\cite{Backus1978}
describes an algebra of programs and is often credited with popularising the
notion of functional programming. His talk directed attention to the work of
Church, Landin and others, inspiring the invention and academic development of a
series of functional languages. In late \decade{1980} a committee was convened
to integrate decades of work in functional programming into a single language:
\emph{Haskell}.

A thorough review of these seminal works will give the reader little idea of
what functional programming looks like today in practice. Sources for this are
numerous, and language dependant, but in general I have found most useful
sources to be instructional material for the Haskell language. Haskell is the
primary language for academic research in the field. A great deal of modern work
involves the use of category theory to develop principled and lawful
abstractions and patterns that improve efficiency, push application boundaries
and ensure safe parallelism and distributed computation.

A book I am fond of is \citeauthor{Allen2016}'s
\citetitle{Allen2016}~\cite{Allen2016}, but there are numerous others which I
have yet to consume. I started with it, because it appears to be one of the more
recent introductory volumes. I have found a useful introduction to category
theory in \citeauthor{Spivak2014}'s \citetitle{Spivak2014}~\cite{Spivak2014}.

A great deal of work in functional programming and \ac{frp} is indexed in the
publications listing of the Yale Haskell group at \begin{center}
\url{http://haskell.cs.yale.edu/publications/} \end{center} The group was
founded by Paul Hudak (\anno{1952}--\anno{2015}) who was a co-founder of
\ac{frp}. The publications are from only one group, but they were very
productive, and cite a lot of relevant literature.

Because I wish to target control applications which often use embedded hardware
that is potentially modest, my work will be implemented in C++. There is very
little formal work on functional programming in C++. A book by Ivan Čukić is in
development for Manning Publications~\cite{Cukic2017} and its expected to be
released later this year. Based on several conference talks from professionals
at companies like Netflix, Facebook, Twitter and Google, there is functional C++
code being written and used.



\subsection{Functional Reactive Programming}


In \anno{1997} Conal Elliott and Paul Hudak published what was to become a
popular paper entitled \citetitle{Elliott1997a}~\cite{Elliott1997a}. In it, they
describe \textit{Fran} (Functional Reactive Animation)\index{Fran (Software)}, a
domain specific language embedded in Haskell for abstracting the composition of
computerised graphical animations. This was the birth of the patterns and
paradigm that would become known simply as
\acf{frp}~\cite{Wan2000,Nilsson2002a}. Since then, \ac{frp} has been ported to
create other domain specific languages. For examples, \textit{FVision} for
visual tracking~\cite{Peterson2001b}, \textit{Frob} for
Robotics~\cite{PeHa99,phh99} and at least 6 other domain specific
implementations of the paradigm.

The \ac{frp} paradigm is a set of abstractions for making time varying signals
and events first class objects. These abstractions are based on a rich substrate
of mathematics and is shown to model hybrid systems exactly in the limit of
small time-sampling. Functional abstractions based on category theoretical
concepts facilitate automatic conversion of non-\ac{frp} functions to work on
behaviours and events. This allows the user to think at a high level of
abstraction, near the mathematical description of elements in the problem and
solution domains.

My focus will be on hybrid systems modeling in distributed environments. In
particular, the research Hudak, Peterson, Nilsson and others. Key works include
\cite{Hudak2003,Nilsson2003,Courtney2003b}. John Peterson specifically addresses
issues of parallelism in~\cite{Peterson2000}. Performance issues that affect
real-time applications are addressed in~\cite{Hudak2003} and references therein.

The richest source of literature I have found on the implementation of \ac{frp}
in C++ is Louai Al-Khanji's Master's thesis,
\citetitle{Al-Khanji2015}~\cite{Al-Khanji2015}. In the thesis he describes a C++
implementation of the paradigm which self-admittedly is lacking in refinement.

There are now several publicly available modern C++ libraries for implementing
\ac{frp}, such as Sodium\SideNote{\url{github.com/SodiumFRP/sodium}},
C++React\SideNote{\url{github.com/schlangster/cpp.react}}[\onelineskip], and
David Sankel's sfrp\SideNote{\url{https://goo.gl/qoqlQP}}[3\onelineskip]. David
Sankel's \ac{frp} implementation has an interesting history. He designed the
program semantics for controlling a robot arm \emph{before} realising that he
had created a functional reactive design. He tells that story in one of his
popular talks from BoostCon
2014\SideNote{\url{youtu.be/tyaYLGQSr4g?t=38m32s}}[\onelineskip]. His entire
personal library, including C++ headers for functional programming and \ac{frp}
are available on his bitbucket repository~\cite{sbase}.


\subsection{ISO/IEC Standard C++}



The history of the C++ language began with its original creator Bjarne
Stroustrup in \anno{1979} as ``C with classes''. The language was not compiled,
but transpiled into C and then compiled using which ever C compiler was
available to the user. It was not until \anno{1998} that the language was
standardised by committee under the \ac{iso} and \ac{iec} and published as
\acl{cpp98}. Between \anno{1998} and \anno{2011}, the C++ language was largely
static, with only a minor update to the standard (with no new features) in
\anno{2003}. A major update in \anno{2011} brought C++ into the modern era,
facilitating styles of programming pioneered by languages such as Python and
Haskell. At the time of this writing, the current standard is \ac{cpp14}; and
\ac{cpp17} is feature frozen and en route to publication in \anno{2017}. The
standards committee currently plans new releases every 3-years, with every other
release considered \emph{major}. \ac{cpp11} was a major release, and \ac{cpp14}
as comparatively minor, but still important. \ac{cpp17} is expected to be the
most drastic departure from \emph{class} C++ (that is, \ac{cpp98}) yet.

Bjarne Stroustrup's compendium, \emph{The C++ Programming
Language}~\cite{Stroustrup2013}, is a very accessible plain language discussion
of the key parts of the \ac{cpp11} standard, with an introduction to the
language in the first chapter. The time constrained reader wanting a primer on
modern C++ in smaller volume will appreciate Stroustrup's \emph{A Tour of
C++}~\cite{Stroustrup2013tour}. That book is basically the first chapter
of~\cite{Stroustrup2013}, augmented so as to be self contained.

Because a core value of the C++ committee is backward compatibility with
previous language versions, the current emphasis of the committee is on making
the language simpler by adding the right features. As early as \anno{1994} (and
probably earlier), Bjarne Stroustrup wrote
%
\begin{quote} Within C++, there is a much smaller and cleaner language
struggling to get out.\\\null\hfill\EpiSource{The Design and Evolution of C++.
p.207} \end{quote}
%
and later added,
%
\begin{quote} And no, that smaller and cleaner language is not Java or C\#\\
\null\hfill\EpiSource{FAQ item ``Did you really say that?'' on
\href{http://www.stroustrup.com/bs_faq.html\#really-say-that}{Dr. Stroustrup's
website}}\end{quote}

In order to evolve the language by adding features without removing any, the new
features must be strategically chosen to bring about shifts in programming
style. New features, used together idiomatically, form replacements for older
techniques, without creating direct redundancies. Code compliant with \ac{cpp98}
will compile on a \ac{cpp14} compiler, but an informed programmer could
distinguish \emph{Modern C++} by inspection, despite the backward compatibility.

Modern code will not only make use of new the new language features, but will
portray a style that has evolved over decades of C++ use by millions of
programmers, computer scientists and software engineers. Such practices are
often learned \emph{on the job}, but are also promulgated in prominent texts,
such as Scott Meyers \emph{Effective Modern C++}~\cite{meyers2014effective}.
However, the most monolithic effort to document and standardise best practices
is the \emph{C++ Core Guidelines} project~\cite{CppCoreGuidelines}. Those
guidelines are supported by the \acl{gsl}~\cite{cppgsl}.

Aside from good general programming practices, lightweight abstractions and
intuitive \acp{api}, the proposed work will focus heavily on developing
thread-safe and parallel algorithms and patterns. The \ac{cpp11} standard was
the first to openly support threading in the language (via the memory model) and
in the C++ standard library. The popular text on that subject is Anthony
Williams' book \emph{C++ Concurrency In Action}~\cite{Williams2012}. The
\ac{cpp14} standard made only small additions to concurrency\SideNote{Such as
\verb|shared\_timed\_mutex|\\ and \verb|shared\_lock|.}. A common criticism of
the C++ standard for concurrency is that the interface for |std::promise| does
not offer composability. This means that they can be processed in order of
completion. Asynchronous event handling mechanisms in C\# and other languages
are composable. The need for composability of synchronisation events has been
widely recognised. The Microsoft Windows \ac{api} has |WaitForMultipleObjects|
and Unix has the venerable |select| call.



\subsection{Optimal Control and Nonlinear Systems}


There are several well established and classic texts in optimal control theory.
I began with an older volume by \citeauthor{Kirk2004} simply titled
\emph{Optimal Control Theory} and subtitled \emph{an
introduction}~\cite{Kirk2004}. There is no emphasis on programming code, but
algorithms for solving optimal control problems are detailed. The book is very
practical and calculation driven. It contains a sturdy introduction to the
calculus of variations. The text focuses on indirect methods using the formalism
of Pontryagin.

Though the calculus of variations is not a necessary for solving optimal control
problems, it is an inseparable part of the field and its literature. The reason
that, historically, so much effort has been put into developing techniques with
this calculus is that it provides extremely useful intellectual abstractions
that aid in analysis and understanding.

A valuable text focusing only on the \emph{Calculus of Variations} is from
Gelfand and Fomin. The original was written in Russian and used as a course
textbook, however the book was translated to English and revised several years
later by Richard A.\ Silverman~\cite{gelfand2000calculus}.

Another important treatise in optimal control is
\citeauthor{stengel1986optimal}'s \emph{Optimal Control and
Estimation}~\cite{stengel1986optimal}. This book, an aged but highly regarded
graduate level text, may easily be the preferred introduction for anyone with a
classical education in (linear) control systems theory. Starting from very
naïve principles, \citeauthor{stengel1986optimal} builds a theory of optimal
control. Developing from that, a theory of estimation which he combines into two
chapters on \emph{stochastic optimal control.}

Despite the fact that this section is dedicated to optimal control, Vidyasagar's
\emph{Nonlinear Systems Analysis}~\cite{Vidyasagar2002} requires special
mention. The analytical techniques it describes are entirely relevant to optimal
control of nonlinear systems. The reader benefits greatly from the intuitive
grasp the author imparts in every page. I fully expect that a good deal of
background in analysis of nonlinear systems in my final thesis will be inspired
heavily from Vidyasagar's writing.

An excellent overview of numerical methods for optimal control is found in a
survey paper by \citeauthor{rao2009survey}~\cite{rao2009survey}. Rao's survey
lacks depth (for practical reasons), but is a valuable index for the original
literature expounding the various techniques. This paper is well paired on a
reading list with \citeauthor{Cannon2004}'s review of efficient algorithms for
\ac{nmpc}~\cite{Cannon2004}. Though \citeauthor{Cannon2004}'s paper is centred
on \ac{nmpc}, and not more broadly on optimal control, his key focus is on
numerical techniques for the minimisation problem which is core to both.



\subsection{NMPC}


As an introduction to the recent state of research, I know of no better source
than David \citeauthor{Mayne2014}'s recent survey~\cite{Mayne2014}. It is
remarkably comprehensive and contains 170 references to books, survey works and
seminal papers in all major sub-fields of model predictive control research. The
paper briefly introduces the core \ac{n/mpc} problem, but is far from an
introduction to the topic. For a more introductory review of \ac{nmpc}, I direct
the reader to the introduction written by Findeisen and
Allgöwer~\cite{Findeisen2002}, or James B.\ Rawlings' tutorial overview in the
June \anno{2000} issue of IEEE Control Systems Magazine~\cite{Rawlings2000}. A
more comprehensive introduction is provided in some of the textbooks in the
field. For example, \citeauthor{Grune2011} in their standard
monograph~\cite{Grune2011}.

The \ac{mpc} technique was first conceptualised several times independently
between the \decade{1960} and \decade{1980}~\cite{Grune2011,Camacho2007}, but
was not able to find widespread use due to the modest computational technology
of those decades—even with completely linear(ised) problems. In the
\decade{1980}, the process control industry, with its characteristically gradual
plant dynamics, took hold of the method following the seminal paper by Richalet
at others~\cite{Richalet1978}. Three decades later, even consumer grade
computing hardware can implement nonlinear-\ac{mpc} for things like unmanned
aircraft~\cite{Eklund2005}, mobile robots~\cite{Teatro2014}, automotive
vehicles~\cite{Abbas2011} and other systems which rapidly traverse their state
spaces. A brief but compelling history of the early development of \ac{mpc} is
given by \citeauthor{Camacho2007} in~\cite{Camacho2007}.

When the process industry embraced \ac{mpc}, the finite horizon meant that no
guarantees of stability existed. It was simply an act of empirically
driven-design that horizon sizes were made large enough to give confidence in
stability. Around \anno{2000}, \citeauthor{Mayne2000} developed the analysis
(using Lyapunov theory) to achieve nominal stability under appropriate
constraints~\cite{Mayne2000}.

Stability in \ac{nmpc} is a difficult problem and an ongoing area of research.
The stability problem stems from the fact that a finite sized preview horizon
cannot provide foresight to navigate around constraint regions, leading to
cyclical integral curves in the vector field (that is, $\fc$ or $\fd$). If the
characteristic size of the constrained regions are small compared to the horizon
size, they do not pose a practical problem for stability. But analytical tools
that can inform our concerns are difficult to wield for non-experts, and those
tools are limited, and often inject undesirable constraints into the problem.

A solution is to employ navigational path planning techniques to plot a desired
route through state-space, and then use the \ac{nmpc} algorithm to track the
desired path. This shifts the responsibility of cycle avoidance away from the
controller and into a higher level planner which can supervise the progress of
the system.



\subsection{Numerical Methods for NMPC}
\label{subsec:lit:numerical-methods}


Conventional approaches to solving \ac{ocpn} can be divided into the categories
of \emph{direct} or \emph{indirect methods}. Indirect methods use the calculus
of variations to determine first-order optimality conditions of the original
control problem, leading to a two-point boundary value problem. Direct methods
call for a discretisation of the problem. Once discretised, it can be
transcribed in a straightforward manner to a \ac{nlp} problem. Many mature,
efficient and well known strategies exist for solving \acp{nlp}.

\citeauthor{Findeisen2002} outline various \ac{nmpc} techniques
in~\cite{Findeisen2002}, as well as Cannon in~\cite{Cannon2004}. More detailed,
but still brisk is Rao~\cite{rao2009survey}.

Both direct and indirect methods will require numerical differential equation
solvers and numerical integrators that can be found in any text on
numerical-methods. However, the approach to finding the solution to the
optimisation problems differ vastly in approach.



\section{Prior Art}%
\label{sec:prior-art}


\textsc{There is no abundance of work} with a focus on software design for model
predictive control systems. As far as I have been able to find, there are no
textbooks, or recent papers.

In \anno{\citeyear{Jobling1994}},\ \citeauthor{Jobling1994} surveyed the impact
of \ac{oo}-programming in control system design~\cite{Jobling1994}. A lot has
changed about the styles and attitudes of software developers since the
mid-\decade{1990}, but those efforts are very much in the spirit of my proposed
work.

Despite the lack of communication about design principles in the control
application, there have indeed been endeavours to make generic software
frameworks for control, such as the \ac{darpa} Software Enabled Control
project\index{Software Enabled Control, \acs*{darpa} project}, or the \ac{nist}
The Real-time Control Systems Architecture.

The \ac{darpa} Software-Enabled Control project~\cite{Keviczky2004,Gill2003}
probably represents the largest independent overlap with my proposed work.
However, some key components of the software are proprietary, and not publicly
available. It was part of a military project, which is made obvious by the many
papers with application to military ground vehicles and aircraft.

A similar project to produce a general framework for control came from the
Intelligent Systems Division of the \ac{nist}\SideNote{Details at\\
\url{https://www.nist.gov/intelligent-systems-division}}. The project, entitled
the \emph{The Real-time Control Systems Architecture} has software support
called the Real-Time Control Systems Library. There is a deep hierarchy of
supportive technologies, such as the \ac{nist} Reference Model Architecture for
Intelligent Systems Design, and the \ac{nist} Neutral Messaging Language, all of
which couple with the library.

Those technologies offered by \ac{nist} are no longer developed or maintained,
and are considered \emph{legacy} projects. They are also remarkably complex
which presents a significant barrier to entry. I am proposing something with a
sufficiently small cognitive footprint that an enthusiastic graduate student
could have a small project running in a week or so.

The software in the components of the \ac{nist} architecture and the publicly
available components of the \ac{darpa} project will be the subjects of my search
for design patterns. While they may be useful as inspiration, I must emphasise
again that I have no intention of developing design tools for control systems.
Rather, I want to develop frameworks to support research and development of new
control techniques based on predictive models.

In \citeyear{Peterson2001a}, \citeauthor{Peterson2001a} produced a case study
using \ac{frp} to control their robot in the \anno{2000} Robocup
competition~\cite{Peterson2001a}. The produced a subset of \ac{frp} using C++
template metaprogramming techniques to create what they describe as a
``powerful, extensible programming environment for robotics''. Some similar
research was produced in \citeyear{Xiangtian2002} by
\citeauthor{Xiangtian2002}~\cite{Xiangtian2002} where they explicitly describe
their work as a \emph{domain specific embedded language} within C++.

With regard to~\ref{contrib:theory}, there is some work done with \ac{nmpc}
controllers which are able to dynamically adjust or swap predictive models
during operation. For example, in \anno{2016} \citeauthor[and related
work]{Zhang2016} developed a computationally aware controller that could vary
the fidelity of the model to reduce computational burden~\cite[and related
work]{Zhang2016}. Also in \anno{2016}, \citeauthor{Ostafew2016} produced a
\ac{nmpc} controller which is able to learn more accurate
models~\cite{Ostafew2016}. The concept of switching components, such as the
predictive model, the minimisation technique and even the constraint techniques
is something I wish to explore, and so there is overlap with these papers.



\section{Literature Gaps}%
\label{sec:literature-gaps}


\textsc{With regard to}~\ref{contrib:design} and~\ref{contrib:capital}, there is
very little literature which directly addresses the problem of engineering the
predictive control systems software. There is a small variety of software
available for designing and implementing fairly straightforward \ac{nmpc}
controllers, but those tools are designed to hide implementation details and
minimise implementation choices, and I am proposing a research framework which
necessarily exposes those decisions.

In the field of software engineering, interest has been building for software
designs using functional and functional reactive designs. The limitations of the
C++ language prior to \anno{2011} made it awkward to produce truly functional
architecture, so progress was limited.

With regard to~\ref{contrib:theory}, there is a lack of good literature
regarding choices in implementation details of \ac{nmpc} controllers. I wish to
study the impact of some of those choices.

Enabled by the highly dynamical nature of the proposed framework, there is a lot
that can be be understood about re-configuring \ac{nmpc} controllers during
operation. This is a fresh and open area of research.

In \Sref{sec:NMPC}, I will describe techniques for discouraging the controller
from certain behaviours using potential fields in the cost function (see
Eq.~\eqref{eq:L-err-u-Phi}). The more direct way to do this is to proscribe
undesired states using constraint techniques in the mathematical
optimisation\SideNote{For example, using Lagrange multipliers, or \ac{nlp}}.
Techniques which penalise behaviour using the cost function, are called
\emph{soft constraints}, while directly proscribed behaviours are called
\emph{hard constraints}.

In textbooks and elementary literature, you will find discussions comparing hard
and soft constraints at a naïve level. There may be discussion of the fact that
hard constraints are computationally more expensive, or that soft constraints do
not provide guarantees against the undesired behaviour. There will be little or
no discussion of techniques for constructing potential fields.



\section{Optimal Control \& NMPC}%
\label{sec:NMPC}


\textsc{Nonlinear \acs{mpc} has}, at its core, a numerical optimal control
problem. In this section I shall focus on theory and methods for solving optimal
control problems as they arise in the context of online \ac{nmpc}.

We will move swiftly into the discrete flavour of optimal control, but the
origins of the problem are rooted in classic and continuous formulations. There
are also several techniques for discretising the problem, so it will be useful
to have a fully continuous case for reference.

Optimal control is quite different from the classical control techniques which
use model information only at design time to choose a rule which maps each state
to a control value to obtain desired behaviour. Optimal control formulates the
control problem in terms of a cost (or reward) functional that maps a state
space trajectory and corresponding control space trajectory to a real number
which quantitatively appraises the paths in terms of desirable behaviour. A
mathematical model of the system is used to estimate a state-space trajectory
that will result from a given control plan. We may then seek the
state-space/control-space trajectory pair which minimise the cost.

A common model for a nonlinear system is
%
\begin{equation}\label{eq:generalNonlinearSystemWithoutInput}
  \dot{\bix} = \fc(t,\, \bix(t)),\quad \bix(0) = \bix_0, \quad t\ge0,
\end{equation}
%
where $\fc : \reals\times\biX\rightarrow \biX$ is a time varying nonlinear map
from a vector of state at time $t$, $\bix(t)\in\biX$, to the state velocity
$\dot\bix$, forming an initial value problem. When \fc\ is specified,
time-integration will yield an integral curve leading from the initial state.
The problem is often visualised with \fc\ representing a vector field (that is,
a velocity field) and the integral curves are the trajectories to which the
vectors are locally tangent at every point.

A controlled system model will
extend~\eqref{eq:generalNonlinearSystemWithoutInput} with a control input, \cu:
%
\begin{equation}\label{eq:generalNonlinearSystem}
  \dot{\q} = \fc(t,\, \q(t),\, \cu(t)),\quad \q(0) = \q_0,\quad t\ge0.
\end{equation}
%
Depending on the specific dependency of \fc\ on $\biu(t)\in \biU$, this
introduction allows us to control the shape of the integral curves or, more to
the point, the shape of the underlying vector field if $\biu$ can be specified
as a function of $\bix$: $\biu(\bix(t))$. The foundational problem of control
systems in this context is to find a law giving the values for the elements of
$\cu$ to ensure that all possible integral curves represent desirable behaviour.

The desirability of a particular solution must be quantifiable as a \emph{cost
function(al),} based on the trajectories through state and control space the
solution represents. For example, if I am trying to find a minimum-time path, a
cost functional could simply be $\int_{t_0}^{t_\mathup{f}}\,\dif t =
t_\mathup{f} - t_0$. But that could (or will) lead to solutions which ride the
limits of the control actuators. Perhaps what you really want is a
\emph{minimum-effort} solution: $\int_{t_0}^{t_\mathup{f}}
\bigl(\cu(t)\bigr)\tr\cu(t)\; \dif t$. Anything with a direct relationship to
the choices of state and control may be penalised in this way. The term in the
integrand responsible for accumulating cost over the trajectory is called the
\emph{running cost}\index{Running cost}, and is usually denoted with a capital
(and sometimes calligraphic) $L$.\SideNote{The custom of using $L$ for the
running cost commemorates the Italian-French mathematician and astronomer
Joseph-Louis Lagrange, who revolutionised physics by reformulating (Newtonian)
mechanics in terms of this sort of functional minimisation.} In many
formulations, it is desirable (for reasons to be discussed) to place an
additional penalty on the location of the terminus of the state-space
trajectory, denoted $e(\q(t_\mathup{f}))$.

\begin{definition}\label{def:ocp}
%
The \emph{\textbf{\acf{ocp}}} is to solve for the free variables $\q(t)$ and
$\cu(t)$ that satisfy the minimisation,
%
\begin{equation}
\min_{\q(t),\ \cu(t)}\; \Biggl[ \int_{t_0}^{t_\mathup{f}} L(t,\, \q(t),\, \cu(t))\,\dif t + e(\q(t_\mathup{f})) \Biggr],\tag{\acs{ocp}}\label{eq:ocp}
\end{equation}
%
subject to
%
\begin{align}
  \q(t=0) - \q_0 &= 0\label{eq:cnt-constraint-init}\\
  \dot{\q}(t) - \fc(\q(t),\, \cu(t)) &= 0\quad t_0 \le t \le t_\mathup{f}\label{eq:cnt-constraint-model}\\
  g_k(\q(t),\, \cu(t)) &= 0\quad t_0 \le t \le t_\mathup{f}\label{eq:cnt-constraint-equality}\\
  h_k(\q(t),\, \cu(t)) &\leq 0\quad t_0 \le t \le t_\mathup{f}\label{eq:cnt-constraint-inequality}\\
  r(\q(t_\mathup{f})) &\le 0.\label{eq:cnt-constraint-terminal}
\end{align}
%
\end{definition}
The constraint~\eqref{eq:cnt-constraint-init} merely ensures that the solution
trajectory starts at the systems known present state.
Equation~\eqref{eq:cnt-constraint-model} ensures that the solution is consonant
with the dynamical description in our model. The
constraints~\eqref{eq:cnt-constraint-equality}
and~\eqref{eq:cnt-constraint-inequality} are algebraic expressions constraining
the solution space. For example, if the control input is limited to a ball in
the control space, it may be expressed here. The final
constraint,~\eqref{eq:cnt-constraint-terminal}, ensures that the terminus of the
solution trajectory will lie in a subset of the state-space. This is key to some
techniques to guarantee stability in \ac{nmpc}. At
minimum,~\eqref{eq:cnt-constraint-init} and~\eqref{eq:cnt-constraint-model} are
part of every well posed \ac{ocp}, while the others are at the discretion of the
designer. Optimal control techniques offer and interesting choice to control
designers, in that constraints may be express in
\emph{hard}\index{Constraints!hard}\index{Hard constraint} or
\emph{soft}\index{Constraints!soft}\index{Soft constraint} varieties. If the
constraint is expressed through active enforcement
of~\eqref{eq:cnt-constraint-equality}–\eqref{eq:cnt-constraint-terminal}, then
it is a \emph{hard} constraint. However, the cost function can be chosen so as
to anathematise state-control configurations without the computational overhead
of hard constraints. This comes at the cost of certainty in the proscription of
such regions, and so these constraints are called \emph{soft.}



\subsection{\acl{nmpc}}%
\index{Nonlinear model predictive control (\ac{nmpc})}


Since our goal is a computer algorithm, we must sample the state and control
trajectories on discrete intervals\index{Sampling (discrete)}. This is true
regardless of how we approach the problem, since it is a limitation imposed by
digital hardware. Choosing a constant sampling interval $\ival$, the system
state at a time $t_k = k\ival$ is notated with the short hand $\bix(k\,\ival) =
\bix^k$. The model from~\eqref{eq:generalNonlinearSystem} takes the form of a
recurrence relation
%
\begin{equation}\label{eq:discreteModel}
  \q^{k+1} = \fd(\q^k,\, \cu^k),\quad\q^0 = \q_0,\quad k = 0,1,2,\ldots
\end{equation}
%
where the \emph{state transition map}\index{State transition map},
$\fd:\stateSpace\times\ctrlSpace\rightarrow\stateSpace$, assigns a successor
state $\q^{k+1}$ from a given state vector $\q\in\stateSpace$ and control vector
$\cu\in\ctrlSpace$. The domains \stateSpace\ and \ctrlSpace\ may be arbitrary
metric spaces, with metrics $d_\stateSpace(\q_1, \q_2)$ and $d_\ctrlSpace(\cu_1,
\cu_2)$. For the time being, it is harmless to imagine that $\stateSpace =
\reals^n$ and $\ctrlSpace = \reals^m$ for $n,m \in \posInts$, with the standard
Euclidean metric $d_\stateSpace(\q_1, \q_2) = \norm{\q_1 - \q_2}$, (and alike
for $d_\ctrlSpace$), but this need not be the case.

Given a \emph{control sequence}\index{Control sequence, $\cu^k$!set of}
$\bigl\{\cu^0,\cu^1,\ldots,\cu^{N-1}\bigr\} \in U^{N}$ for $N\in\posInts$, we
can forecast a corresponding (discrete) state-space trajectory $\qcu^k$ based on
the model. Starting from an initial position $\q^0$, simply
iterate~\eqref{eq:discreteModel} recursively, using a previously obtained value.
That is, starting from $\q^0$ we obtain $\q^1 = \fd(\q^0, \cu^0)$, and then
$\q^2 = \fd(\q^1, \cu^1)$, and so on until $\q^N = \fd(\q^{N-1}, \cu^{N-1})$. In
this context, the integer $N$ is referred to as the \emph{prediction
horizon}\index{Horizon, prediction}, or simply as \emph{the horizon.}

\begin{definition}\label{def:ocpn} The \emph{\textbf{\acf{ocpn}}} is to solve
for the free variables in the  sequences
%
$\q =\bigl\{\q^0,\q^1,\ldots,\q^{N-1}\bigr\}$
%
and
%
$\cu =\bigl\{\cu^0,\cu^1,\ldots,\cu^{N-1}\bigr\}$
%
that satisfy the minimisation,
%
\begin{equation}
  %
  \min_{\substack{\q\in\stateSpace^{N+1}\\ \cu\in\ctrlSpace^{N+1}}}\; \Biggl[ \sum_{k=0}^{N-1} L(\q^k,\, \cu^k) + e(\q^N) \Biggr]\tag{\acs{ocpn}}\label{eq:ocpn}
  %
\end{equation}
%
subject to
%
\begin{align}
  %
  \q^0 - \q_0 &= 0\\
  \q^{k+1} - \fd(\q^k, \cu^k) &= 0\quad k = 0,\ldots,N-1 \label{eq:ocpn-constraint-model}\\
  g_k(\q^k, \cu^k) &= 0\quad k = 0,\ldots,N-1\\
  h_k(\q^k, \cu^k) &\leq 0\quad k = 0,\ldots,N-1\\
  r(\q^N) &\leq 0
  %
\end{align}
%
\end{definition}
Here, the constraints play the same roles (now in discrete form) as they did for
Definition~\ref{def:ocp}, the definition of the \ac{ocp}.

It is not uncommon for Definitions~\ref{def:ocp}~and~\ref{def:ocpn} to be
augmented with an algebraic problem to be solved simultaneously with the
differential problem. Such problems have an additional set of free variables for
the algebraic problem. It adds complexity without insight, and is not pertinent
to the chosen examples, it is not discussed further.

The objective of the minimisation in \ac{ocpn} is our discrete cost
function\index{Cost function} which is commonly notated as
$J_N:\stateSpace^N\times\ctrlSpace^N\rightarrow\reals$. Because the model
constraint~\eqref{eq:ocpn-constraint-model} unambiguously maps a control
sequence to a state-space trajectory, a large class of numerical methods for
\ac{nmpc} eliminate the state-space variables and free variables in the
minimisation problem. For those methods,~\eqref{eq:ocpn} might be rewritten as
\begin{equation} \label{eq:u_optimal} \ou = \argmin_{\cu\in\ctrlSpace^N}\,
J_N(\qcu,\, \cu). \end{equation} The ornamental `$\star$' denotes values
relating to the solution of the optimisation problem. So \ou\ is the optimised
control sequence (that is, the one which minimises the cost function), while
\oq\ would represent the corresponding trajectory through state-space.

In the \ac{nmpc} algorithm, \ou\ is computed repeatedly and the first element,
$\ou^0$, is executed before solving the problem again with a newly measured
$\q^0$. A useful notational tool is the feedback law: $\cu(t) = \fbk(\bix(t))$.
In this context, the feedback law $\fbk_c:\stateSpace\rightarrow\ctrlSpace$ maps
the current state to the appropriate control\SideNote{If the feedback law can be
expressed in closed form, then simple substitution, \[ \fc(t, \bix(t),
\fbk_c(\bix(t))) = \fc^\prime(t, \bix(t)), \] can
transform~\eqref{eq:generalNonlinearSystem} back
to~\eqref{eq:generalNonlinearSystemWithoutInput}, justifying my comment that
choosing a control law can be thought of as manipulating the shape of the
underlying vector field.}. Let us summarise the \ac{nmpc} algorithm more
precisely: \begin{displayAlgorithm}[The general \ac{nmpc}
algorithm]\label{alg:NMPC-general}\leavevmode%
%leavemode avoids a bug in amsthm with theorems starting with lists.
%
  \begin{algorithmic}[1]
    \Repeat\Comment{NMPC loop}
    \State Measure or estimate the state $\q^j\in\stateSpace$
    \State Set $\q_0 = \q^j$ and solve \ac{ocpn} for those initial conditions
    \State Define the feedback control value $\fbk(\q^j) \coloneq \ou^0\in\ctrlSpace$ and execute
    \Until{control process is inactivated}
  \end{algorithmic}
\end{displayAlgorithm}



\subsubsection{The Cost Function}%
\index{Cost function}


The choice of the cost function impacts the behaviour and performance of the
controller in the most thorough sense. The cost function is how the designer
expresses the notions of good and bad behaviour. But since this function is core
the optimisation problem---the most computationally intensive portion of the
controller---a balance is to be struck between a thorough quantification of
behavioural characteristics and the practical performance of the controller.

The cost function\index{Cost function} is commonly expressed as a sum of
\emph{running cost terms}\index{Cost function!Running cost} over the horizon,
with an additional term depending only on the terminal state, called the
\emph{terminal cost}\index{Cost function!Terminal cost}:
%
\begin{equation}\label{eq:costfn-aliased}
  J_N(\q,\, \cu) \coloneq \sum_{k=0}^{N-1} L(\q^k,\,\cu^k) + e(\q^N).
\end{equation}



\subsubsection{The Running Cost}%
\index{Cost function!Running cost}


The function $L:\stateSpace\times\ctrlSpace\rightarrow\reals$ appearing
in~\eqref{eq:costfn-aliased} is the \emph{running cost}\index{Running cost} and
expresses the operational costs of traversing state-space-time.

Without loss of generality, the controllers described here are designed to track
a reference path in state space:\index{Running cost!tracking error}
%
\begin{equation*}
  \q\rnc=\bigl\{\q\rnc^0,\ \q\rnc^1,\ldots,\q\rnc^{N-1}\bigr\}.
\end{equation*}
%
To track the above path is to minimise the error function $\qerr = \q\rnc -
\q_\cu$\index{Error function}. This desire quantified quadratically in the
running cost function as
%
\begin{equation}\label{L-err}
  L\bigl(\qcu^k,\cdot\bigr) \coloneq \bigl(\qerr^k\bigr)\tr\mQ\,\qerr^k
\end{equation}
%
where \mQ\ is positive-definite matrices of weighting coefficients. The
positive-definiteness of \mQ\ ensures that, all other things being equal,
$\norm{\q_1} > \norm{\q_2} \iff L(\q_1,\cdot) > L(\q_2,\cdot)$.

If the the elements of the reference sequence are all equal (that is to say the
reference is static) then the tracking controller becomes a set-point
controller.

Minimising the tracking error in the way described, surprisingly to some, may
lead to very undesirable behaviour. Plain error minimisation without further
regard will drive actuators to their limits because the \emph{only} desire
expressed in formulation is adherence to the reference trajectory. Even if you
constrain your optimisation to respect the limits of your actuators, the
controller will tend to extremes—casually running at the operational limits you
prescribe in the constraints.

Solutions which balance the tracking error against the control effort must be
explicitly called for.\index{Running cost!control effort} A quadratic penalty on
control effort, in addition to the tracking error terms will balance needless
aggression in the controller against tracking accuracy. (Assuming the notion of
\emph{needless} aggression means anything in your application.)

An individual element in the control sequence is, in general, a time varying
function $\cu^k(t)$ with a duration equal to \ival. Since the cost and
cumulative effect of a control is computed by integration, $\cu^k(t)$ should
locally Lebesgue integrable. Let \lebesgueIntegrable{\Omega}[s] denote the set
of all locally Lebesgue integrable functions that map $\Omega$ to $\reals^s$.
Then the control space\index{Control space, \ctrlSpace} is expressed as
$\ctrlSpace = \lebesgueIntegrable{\intco{0}{\ival}}[m]$. In such cases, the
control effort penalty would take the form,
%
\begin{equation}\label{eq:L-integral-u}
  L(\cdot,\,\cu^k) \coloneq \cdots + \frac{1}{\ival}\int_0^{\ival}\bigl(\cu^k(t)\bigr)\tr\mR\,\cu^k(t)\,\dif t.
\end{equation}
%
where, as it was with \mQ, the matrix of coefficients, \mR, should also be
positive definite.

I previously wrote that the reader may harmlessly think of \ctrlSpace\ as simply
$\reals^m$, and the attentive reader may be wondering about the apparent
dissonance I am creating with this discussion of \ctrlSpace\ as a function
space. This tension is broken when we implement the so called \emph{zero-order
hold}\index{Zero-order hold} for the control sequences, in which controls are
held constant between sampling nodes\SideNote{If the sampling interval is
sufficiently short, then the zero-order hold is a very practical choice to make.
More elaborate choices are indicated when the sampling interval is longer, or
the physics of the underlying mechanism do not allow for constancy between
sampling nodes.}. The choice eases mathematical rigour and reduces the
computational burden of the final algorithm. The set of all constant functions
mapping the interval $\intco{0}{\ival}$ to $\reals^m$ are a subset of
\lebesgueIntegrable{\intco{0}{\ival}}[m], so the choice is algebraically sound.
Under the zero-order hold condition, the contribution of the control effort to
the running cost simplifies from~\eqref{eq:L-integral-u} to
%
\begin{equation}\label{eq:L-u}
  L(\cdot,\cu^k) \coloneq \cdots + \bigl(\cu^k\bigr)\tr\mR\,\cu^k.
\end{equation}
%
While the zero-order hold is a very common convention, the explorations of the
section on \emph{numerical methods of \ac{nmpc}}, \Sref{subsec:lit:numerical-methods},
will describe alternative ways to discretise the problem.

\sAsterism

\textsc{In most} applications of \ac{nmpc}, the running cost consists only of the terms described in Equations~\eqref{L-err} and~\eqref{eq:L-u}. In summary,
%
\begin{equation}\label{L-err-u}
  L(\qcu^k,\,\cu^k) \coloneq \bigl(\qerr^k\bigr)\tr\mQ\,\qerr^k + \bigl(\cu^k\bigr)\tr\mR\,\cu^k.
\end{equation}

In some applications, it is useful to disproportionately discourage paths
through a particular portion of state-space. Take, as an example, the case of a
self-driving car which may want to move slowly near blind corners (thus
discouraging states of high velocity near those areas). These regions of
state-space may be cast as inauspicious with a scalar field\index{Running
cost!soft constraints}\index{Running cost!scalar field} which is accumulated in
the running cost. The field $\Phi:\stateSpace\rightarrow\reals$\index{Cost
function!Potential field term} takes on high (positive) values in regions of
$\stateSpace$\ which the designer wishes to ward against. If $\Phi$ is finite,
it does not preclude trajectories from entering those regions with large $\Phi$,
but it does discourage such trajectories as solutions of cost minimisation. We
therefore call such constraints \emph{soft constraints.}\index{Constraints!soft}
This brings us to our final and most comprehensive formulation of the running
cost,
%
\begin{equation}\label{eq:L-err-u-Phi}
  L(\qcu^k,\,\cu^k) \coloneq \bigl(\qerr^k\bigr)\tr\mQ\,\qerr^k + \bigl(\cu^k\bigr)\tr\mR\,\cu^k + \Phi(\q^k).
\end{equation}
%
the value of which is increased by deviations from a reference path, non-zero
control action and placement in state space regions characterised by large
values of $\Phi$.



\subsubsection{The Terminal Cost}%
\index{Cost function!terminal cost}


The terminal cost (or terminal penalty) from~\eqref{eq:costfn-aliased} is the
function $E\mathcolon\stateSpace\rightarrow\reals$ which depends only on the
terminal state $\q^{N-1}$. It is used to insinuate the trajectory towards a
particular state. The term conventionally shares the quadratic form with the
error and control penalties in the running cost. Specifically,
%
\begin{equation}\label{eq:terminal-cost}
  e(\q) \coloneq (\biy-\q)\tr\mS\,(\biy-\q)
\end{equation}
%
with  $\biy\in\terminalSet\subseteq\stateSpace$ and weighting matrix \mS\
sharing the same properties as \mQ. A lot of stability analysis of \ac{n/mpc}
centres around choosing the terminal set \terminalSet, and the terminal target
\biy\ need not have any relationship to $\q\rnc$.

It should be superficially clear that the terminal cost in  term is minimised
when the optimal state trajectory \oq\ ends a closely as possible to \biy. It is
useful to further appreciate that since \oq\ is constrained by the state
succession relationship, the influence of this term is propagated backward
through the trajectory during the minimisation. Speaking qualitatively, this
gives the appearance of a somewhat rigidly connected set of points forming the
state trajectory. If one were to pinch the end of the ridged line and push it
around, the path would give some hindrance as the effect of the constraint
resists forbidden motions.\SideNote{Later on, we will see that this effect I
characterise as rigidity, is actually the influence of Lagrange multipliers
working to constrain the system during the mathematical optimisation.}

Traditionally, \ac{n/mpc} schemes with guaranteed stability for nonlinear
systems impose conditions on the allowed regions for terminus of the state-space
trajectory, or other such demanding hypotheses on the system which make the
on-line computation of the open loop optimal control difficult. Softly
constraining the terminus with the quadratic error penalty eases the
calculations, but sacrifices guarantees of stability.

  %!TEX root = Proposal-PhD.tex
%
\chapter{The Proposal}%
\label{chap:proposal}


In the introductory chapter, \Sref{chap:introduction}, I enumerated three
primary contributions of my thesis:
%
\MarginDefinition{A \textbf{\textit{\acl{sdd}}} is, as outlined in \ac{ieee1016}
is a description of a software product that \textit{can be produced to capture
one or more levels of concern with respect to its design subject. These levels
are usually determined by the design methods in use or the life cycle context;
they have names such as “architectural design,” “logical design,” or “physical
design.} I will use the term to describe, mainly, the concerns of what is done
within the software components and their interrelationships, as opposed to
\emph{how} anything is done.}
%
\begin{enumerate}[label=(C\arabic*)]
  %
  \item the \textbf{\textit{\acl{sdd}}}. \label{contrib:design}
  %
  \item the \emph{software capital}, and \label{contrib:capital}
  %
  \item novel theoretical research in the field of optimal control
  systems.\label{contrib:theory}
  %
\end{enumerate}
In this chapter, I shall attempt to itemise and prioritise some milestones,
principles and guidelines for the proposed research. It is important to keep in
mind that, at any given stage of research, the forward direction will be
determined by intermediate findings. At this point, it is impossible to predict
those directions, and making plans that are too specific may, at best, be a
misuse of attention and energy. So, I will focus on describing, in general
terms, parameters and values that will guide and inform my decisions along the
way.



\section{Research Objectives for the Software Design}


\textsc{Item} \ref{contrib:design} in my list of contributions is the
architectural description of the software framework, the \ac{sdd}. It can, in
places, be difficult to decouple \ref{contrib:design} from
\ref{contrib:capital}, which embodies the actual code. The architecture consists
of descriptions that comport with \acl{ieee1016} (and possibly
\acl{ieee420101}), and can be implemented in any \emph{Turing complete}
language. But the language can have a large impact on the form of patterns and
contracts.

For example, in a strongly typed language like C++ or Java, interface
specifications are an important component of polymorphism. In \emph{duck typed}
languages like Python, interfaces are meaningless, and inheritance is only used
to share implementation between classes.

The \ac{sdd} will outline the software components, their contractual obligation
to the user and their interrelationships. However, the design of any given
component should be flexible and open to easy modification and adaptation. So, I
expect that some of the strongest contributions will be in the form of patterns
and reusable design solutions for common problems in control systems software.


\section{Research Objectives for the Software Capital}%
\label{sec:ObjectivesForSoftwareCapital}


\begin{flushright}
\begin{minipage}{0.85\textwidth}
  \footnotesize
  {\color{git-ltgray}
  \begin{verbatim}
    Python 3.5.2
    Type "help", "copyright", "credits" or "license" for more information.
  \end{verbatim}
  }\null\vskip-3\baselineskip
  {\ttfamily
  \noindent {>}{>}{>} \textcolor{git-red}{import} this
  }
  \begin{verbatim}
    The Zen of Python, by Tim Peters

    Beautiful is better than ugly.
    Explicit is better than implicit.
    Simple is better than complex.
    Complex is better than complicated.
    Flat is better than nested.
    Sparse is better than dense.
    Readability counts.
    Special cases aren't special enough to break the rules.
    Although practicality beats purity.
    Errors should never pass silently.
    Unless explicitly silenced.
    In the face of ambiguity, refuse the temptation to guess.
    There should be one-- and preferably only one --obvious way to do it.
    Although that way may not be obvious at first unless you're Dutch.
    Now is better than never.
    Although never is often better than *right* now.
    If the implementation is hard to explain, it's a bad idea.
    If the implementation is easy to explain, it may be a good idea.
    Namespaces are one honking great idea -- let's do more of those!
  \end{verbatim}
\end{minipage}\\
\EpiName{Tim Peters}
\EpiSource{The Zen of Python}
\end{flushright}

\noindent\textsc{I open this section with a poem} by the great Tim Peters, the
software engineer and computer scientist of \texttt{timsort}\SideNote{timsort is
the algorithm used in the standard sorting commands in Python, Java SE 7, on
Android and in GNU Octave.} fame. (And if you did not know about it, yes, you
can import the |this| module in python, and it prints out Peters' poem.) The
poem, called \emph{The Zen of Python}, expresses some values of software
craftsmanship that comport well with the goals of the software component of the
proposed research. Those values largely emphasise readability and clarity over
clever shortcuts.

As I learn and understand more about functional programming, I realise that
there may be value in code that is \emph{hard to explain}, seemingly in
contrition to Peters' dictum. But the difficulty in explaining functional
designs is not from needless complexity. It comes from a cultural barrier. As
students, we simply are not taught to think and understand functional patterns
in programming and one incurs that overhead, at least once, when seeing it for
the first time. Once someone has overcome the learning curve, functional designs
can be very clean and understandable when compared to equivalent procedural
code. (For a practical but anecdotal investigation, see the \ac{darpa} sponsored
study by \citeauthor{Haskell-vs-ada} in
\citetitle{Haskell-vs-ada}~\cite{Haskell-vs-ada}.)

The \ac{oo}-paradigm organises code to encapsulate the moving parts. Functional
code reduces the number of moving parts. When implemented puritanically, neither
of these paradigms is a panacea. The best way to think about predictive control
systems is certainly found in a balance between these two philosophies. I wish
to explore that balance and describe some rules and patterns which merge the
best from each.

Regardless of the principles I uncover in the course of the proposed research,
the following list of design goals will remain at high priority:
%
\begin{description}
  %
  \item[Compliant.] It should comply strictly with \acl{cpp17} and the C++ Core
  Guidelines~\cite{CppCoreGuidelines}\SideNote{From the GitHub project
  description: \emph{The C++ Core Guidelines are a set of tried-and-true
  guidelines, rules, and best practices about coding in C++}}. This will include
  use of the \ac{gsl} wherever appropriate.
  %
  \item[Practical.] Code which is high quality and production ready. It should
  be efficient enough for real-world use. It should be straightforward to use,
  read and maintain. It will conform to a formatting style that optimises for
  readability. My metric for this item will be this question: \emph{can a clever
  and well motivated graduate student get a project up-and-running in less than
  a week?}
  %
  \item[Unit Tested.] The entire codebase, including examples, should be covered
  by unit tests. If a user begins a new design based on the example code, they
  can adapt the tests as needed. Of course, \SI{100}{\percent} test coverage is
  a rarely achieved goal, but it is always a good goal to have.
  %
  \item[Lightweight.] The framework itself should be lean enough that a control
  system designer can, if they are able to cope with the numerics, deploy the
  framework on modest hardware, such as Arduino, Raspberry PI, Intel Edison, or
  TI Beagle Bones. Emphasis will be placed on abstraction that is zero-overhead
  or low-overhead, but never at the cost of clarity.
  %
  \item[Portable.] It should compile with the latest \ac{gcc} C++ compiler and
  Clang++ compilers. The code will compile without warnings using the
  |--pedantic| compiler flag.
  %
  \item[Pattern-oriented.] Emphasis is placed on using patterns in order to make
  the code more easily understood. The framework needs to be readily extensible
  by control systems researchers. The use of patterns makes the code more
  expressive, and suggests good techniques and practices to those extending the
  code.
  %
  \item[Minimal mutable state.] In the context of software, a mutable state is a
  feature of routines that store objects and change values during runtime.
  Whenever you write code with an assignment outside of initialisation, you are
  modifying state. I am relatively certain the product of my research will not
  be without mutable state. But there are good reasons to avoid needless use of
  state.

  Mutable state introduces complexity and opportunities for bugs, especially in
  parallel applications. Control systems researchers who are extending the code
  should not have to worry about subtleties of the framework obstructing their
  research. Where mutable state is sufficiently sensible, I will attempt to
  abstract it away from the \ac{api}, so it does not complicate parallelism or
  distribution of the computational tasks.

  Truly stateless programs are a feature of \emph{pure functional programming}.
  However, it is not practical as a public solution, and a major part of the
  proposed research is discovering a balance that is practical within the
  context of modern C++ and the control systems community.
  %
\end{description}

The code will be documented, and a suite of examples will serve as illustrative
documentation, templates or starting points for users. Those examples will
include common problems from optimal control textbooks. Some of the selected
textbook questions include:
%
\begin{itemize}
  \item \ac{pid} control of a damped, driven harmonic oscillator
  \item \ac{nmpc} control of a train-like vehicle with throttle and break control.~\cite[p.~5]{Kirk2004}
  \item \ac{nmpc} control of artificial satellite altitude adjustment, as in~\cite[p.~1]{Vinter2010}
\end{itemize}



\section{Research Objectives in Control Systems Theory}%
\label{sec:ObjectivesForControlSystemsTheory}


\textsc{The following subsections} each describe (briefly) an avenue for
research in the theory of optimal control and \ac{nmpc}, with a view
toward~\ref{contrib:theory}. However, many of these projects also provide sample
code and documentation for contributions~\ref{contrib:design}
and~\ref{contrib:capital}.


\subsection{Analysis of Hard and Soft Constraints}

While designing the controller for virtualME\SideNote{For details on the
virtualME controller, see \Aref{chap:Pubs-Robotics}.} I created a useful
software design for handling artificial potential fields. The software still
needs refinement and extension before it is complete.

One of my interest lies in analysis and design of artificial potential fields,
and the possibility of switching constraints between hard and soft flavours
during runtime. Since soft constraints are computationally lighter, this can be
a route to increasing efficiency and reliability.


\subsection{Robotic Reflexes}

With the \ac{nmpc} algorithm in dynamic state spaces, or with dynamically
configuration artificial potential fields, there is always a concern that the
optimisation step may take too long, when the system must react very quickly to
avoid disaster. Search based optimisation strategies are faster when the
starting \emph{guess} is close to the optimum. The techniques for providing
better guesses is called \emph{hot-starting}.

I would like to try using a neural network to give a robotic system a behaviour
which resembles our human reflexes. The artificial potential field around the
robot can be sampled on a mesh. That mesh can be used as in input to the neural
network, which can either hot-start the optimisation step, or even circumvent
the optimisation step providing a suitable approximation to the \ac{nmpc}
solution.


\subsection{Comparison Among Gradient Based Search Strategies}

When I was struggling with the original \ac{c99} controller, there was a study I
was particularly interested in performing. The framework I am designing would
make it nearly trivial.

In this study, I would like to extend my already existing gradient based
controller (see \Aref{chap:Pubs-Robotics}) with a new optimiser. I would like to
build an \emph{adaptor}-class\SideNote{See the \emph{adaptor pattern} in the
\ac{gof} book~\cite{GOF}.} for the \ac{gnu-sl} optimisation routines. It will
then be trivial to switch between them.

It would be interesting to explore which routines may be most effective when
paired with certain types of artificial potential field. This relates to the
analysis of hard and soft constraints, described in the previous subsection.


\subsection{Comparison Between Gradient and \acs*{nlp} Optimisation}

The choice between gradient based and \ac{nlp} based algorithms is one faced by
control designers each time they implement an \ac{nmpc} controller. Despite that
fact, there seems to be a lack of literature comparing the two in a rigorous
way.

It is a near certainty that one technique will have advantages over the other in
certain conditions. One of my research curiosities is in the possibility of
switching between the two representations of the problem, leveraging the
advantages of each.


\subsection{Dynamic model adaptation}

Functional software design makes it extremely natural to compose functions. This
means that optimisation techniques such as \emph{genetic algorithms} are
relatively straightforward to implement.

I would like to design a supervisory algorithm which evolves the predictive
model, to learn, improve and adapt the model under dynamic conditions. This
research parallels the work done by \citeauthor{Bongard2006} in
\cite{Bongard2006}.



\section{Sequence of Research Milestones}


%
\begin{enumerate}
  %
  \item Once I have identified relevant patterns, I can continue designing the
  final architecture. The patterns will provide a layer of communication through
  nomenclature, while the more ridged components are expressed through
  contracts.
  %
  \item In the next phase, common textbook example questions will be implemented
  as examples for use. Since the solutions to these problems are well known and
  documented, they are good tests. And, their implementation as example code is
  a useful form of documentation that allows users to focus on implementation
  without the additional complexity of an unfamiliar problem.

  Some of the selected textbook questions include:
  %
  \begin{itemize}
    %
    \item \ac{pid} control of a damped, driven harmonic oscillator
    %
    \item \ac{nmpc} control of a train-like vehicle with throttle and break
    control.~\cite[p.~5]{Kirk2004}
    %
    \item \ac{nmpc} control of artificial satellite altitude adjustment, as
    in~\cite[p.~1]{Vinter2010}
    %
  \end{itemize}
  %
  \item At this stage, I should reasses the research avenues
  for~\ref{contrib:theory}, (as described
  in~\Sref{sec:ObjectivesForControlSystemsTheory}), and determine an appropriate
  order for implementing those.
  %
  \item In the final stage of development, I want to re-implement the virtualME
  controller and augment it with an algorithm to learn and improve the system
  model, as described in~\Sref{sec:ObjectivesForControlSystemsTheory}.
  %
\end{enumerate}
%



\section{Approach and Methodology}


I am developing software to further my own research goals, as
per~\Sref{sec:ObjectivesForControlSystemsTheory}. I have also laid out in
several simple example problems as part of the documentation and testing. I have
outlined comprehensive unit-test coverage as a goal in~
\Sref{sec:ObjectivesForSoftwareCapital}.

This is an ideal situation for a tight cycle of development driven by test-first
principles, such as the \acf{tdd} model of \citeauthor{Beck2003}, as expounded
in his \citeyear{Beck2003} book \citetitle{Beck2003}~\cite{Beck2003}. The
\ac{tdd} method has proven very effective, and is often considered to be a
component of the \emph{Agile software development} and \emph{Extreme
programming} development methods, and others.

The benefits of \ac{tdd} include
%
\begin{itemize}
  %
  \item allowing us to catch bugs before the cause problems,
  %
  \item future-proofs against new bugs in an evolving codebase
  %
  \item projects an image of dependability and a mechanism for proving it
  %
  \item facilitates safe re-factoring when users adapt it to their needs.
  %
\end{itemize}

There are many popular books prescribing \ac{tdd}, such as
\citeauthor{Martin2009}'s \citetitle{Martin2009}~\cite{Martin2009}, or the
aforementioned \citeauthor{Beck2003}~\cite{Beck2003} or his later
\citetitle{Beck2007}~\cite{Beck2007}. But most importantly, there are also
academic works advocating for the method. For example, the HydroShare case study
in \citetitle{Carver2016} by \citeauthor{Carver2016}~\cite{Carver2016}, or the
book \citetitle{Kandt2006} by \citeauthor{Kandt2006}~\cite{Kandt2006}.
\citeauthor{Kandt2006}'s book expounds the use of \emph{extreme programming}, of
which \ac{tdd} is an integral part.

Most pertinently, a \anno{2005} study commissioned by Canada's National Research
Council found that \ac{tdd} can lead to higher productivity~\cite{Erdogmus2005}.
In particular, they conclude
%
\begin{quote}
  %
  Our main result is that Test-First programmers write more tests per unit of
  programming effort. In turn, a higher number of programmer tests lead to
  proportionally higher levels of productivity. Thus, through a chain effect,
  Test- First appears to improve productivity. We believe that advancing
  development one test at a time and writing tests before implementation
  encourage better decomposition, improves understanding of the underlying
  requirements, and reduces the scope of the tasks to be performed.
  %
\end{quote}

I also want to investigate the fusion of \acf{dbc}~\cite{Meyer1992} with
\ac{tdd} in C++. The \ac{cpp17} standard has no native support for contract
specification, though there are some implementations where some have adapted
|assert| for those purposes. However, I think \ac{tdd} provides a mechanism to
define and test contractually defined behaviour. Each \emph{precondition},
\emph{postcondition} and \emph{invariant} can be the subject of a unit-test.

  %!TEX root = Proposal-PhD.tex
%
\chapter{Conclusion}%
\label{chap:conclusion}

\textsc{The complexity of control systems} at and beyond the state of the art is
producing proportionally complex software. Management of software complexity is
an ongoing issue within the software engineering industry. New software
development techniques and architectural styles are being practiced to improve
software quality as a means of controlling complexity.

The preceding document has outlined a course of research expanding modern
control techniques \emph{and} software designs that enable more rapid progress
is \ac{nmpc} control systems research. I desire to take lessons from the leading
edge of the software industry and from computer science to discover techniques
and best practices for designing software for \ac{nmpc} that is scalable,
distributable and parallelisable.

For the proposal, I outline three main areas of contribution:
\begin{enumerate}[label=(C\arabic*)]
  %
  \item the \acl{sdd}.
  %
  \item the software capital, and
  %
  \item novel theoretical research in the field of optimal control
  systems.
  %
\end{enumerate}

In a test-driven development process, I wish to develop a modern framework using
functional, reactive programming techniques and the modern \ac{cpp17} language.
The product of that endeavour covers the first two contributions.

The proposed framework is to be the basis for several research projects,
\textit{in potentiâ}. For example, one such project investigates the use of
artificial neural networks to accelerate the optimisation stage of the
algorithm, based on a sampling of the potential field which simulates obstacles
in state space. That project, and the others hold novelty in the field of
control systems research and represent progress in \ref{contrib:theory}. They
additionally demonstrate the usefulness of the framework which underlies them.

  \clearpage
  \appendix
  \renewcommand{\theequation}{\oldstylenums{\Alph{chapter}.\arabic{equation}}}
  \chapterstyle{default}
  %!TEX root = Proposal-PhD.tex
%
\chapter{A Brief Discussion for the Purpose of Establishing Notation}%
\label{chap:notation}
\epigraph{
  We could, of course, use any notation we want; do not laugh at notations; invent them, they are powerful. In fact, mathematics is, to a large extent, invention of better notations.
}{
  \EpiName{Richard P.\ Feynman (1918--1988)}
  \EpiBio{The \emph{great} physicist, my personal hero}
}

\noindent \textsc{Unless otherwise} indicated, the mathematics in this document
is typeset with strict adherence to \ac{iso-math}\index{ISO/IEC 80000-2:2009}.
Undocumented deviations should be considered a bug. To save the reader the
trouble of finding and reading the standard, I shall summarise herein with
particular care to point out anything not in common convention in North America.

Quantities which are not variable across time or context (such as immutable
constants of mathematics) are set in upright text. For examples, $\exp(1) =
\mathup{e}$ and not the italic $e$; the ratio of a circle's diameter to its
circumference (in flat space) is $\symup{\pi}$ and not $\pi$; the imaginary
unit, defined by $\iu^2=-1$, is not the italic $i$, and so on. Variables,
parameters, contextual constants, running numbers  and alike are set in
italicized text. For example, $\sum_i a_i = 2+4\iu$ where $i$\/ is a counter
while \iu\ is the imaginary unit.

This rule of italics vs.\ upright is universally conventional for functions.
This is why $\sin(x)$ is correct and the commonly seen italic $sin(x)$ is widely
recognized as an error. (Some \LaTeX\ users, particularly those who tend to
forget backslashes such as the one in \verb|\sin(x)|, are common offenders.)

Vector and matrix quantities follow the convention of italics vs.\ upright, and
are notated in bold. In addition to being bold, it will generally (but not
necessarily) be the case that vectors will be lowercase, while matrices are
uppercase:
\begin{equation}\label{eq:transpose}
  (\mathbfit{Ax})\tr\,\mathbfit{y} = \mathbfit{x}\tr\,(\mathbfit{Ay}).
\end{equation}
(Note, this is the foundational identity of the matrix-transpose of
$\mathbfit{A}$. That is, the matrix $\mathbfit{A}\tr$ that makes the Left Hand
Side (LHS) inner product equal to the Right Hand Side (RHS) inner product, for
all suitably shaped $\mathbfit{x}$ and $\mathbfit{y}$, is by definition the
transpose of $\mathbfit{A}$.)

The $n\times n$ identity matrix is a matrix quantity, (so is set in bold and
uppercase), but it is defined for matrices independently of context. As such,
its symbol should be bold and upright: \eye{n}\SideNote{This marks a
deviation from the \acs*{iso}/\acs*{iec} standard which discloses a bold
italicised $\mathbfit{I}$ for this purpose. I view this as an oversight in the
standard since it departs from the convention of printing mathematical constants
in Roman. I have found good examples where others make this same
deviation.}[-2.5cm].

I find the following convention to be the most awkward to North American eyes. I
have had reviewers mistake it for poor typesetting despite the great care I take
with my documents. \textit{In following with the convention that context
independent objects are upright, the differential operators are typeset
upright.}\/ For example:
\[
  \diff{f}{x} = \partial_x f = \expe^{\iu x}  \implies f(x) = -\iu\,\expe^{\iu x}+c,
\]
not,
\[
  \frac{df}{dx} = \symit{\partial}_x f = \expe^{\iu x}  \implies f(x) = -\iu\,\expe^{\iu x}+c.
\]
The differences between the upright partial $\partial$ and \textit{italic}
partial $\symit{\partial}$ seem trivially subtle. But the upright infinitesimal
$\dif x$ avoids confusion with the common notation for a distance metric
$d(\mathbfit{x}, \mathbfit{y})$. The expression $d d/d\mathbfit{x}$ is nonsense
while $\dif d/\dif\mathbfit{x}$ is at least meaningful, even if it is
distracting.

A good notation should do some work for you. It frees the mind to bring full
attention to bare on the real problem, and even carry you part of the way. The
legendary mathematician Bertrand Russell is reputed to have said that \textit{a
good notation has a subtlety and suggestiveness which at times make it almost
seem like a live teacher.}

In the spirit of the previous paragraph, I feel compelled to comment that I
dislike this dot: $\mathbfit{a}\cdot\mathbfit{b}$ in the context of the vector
\emph{dot} product. The great applied mathematician William Gilbert ``Gil''
Strang even calls it unprofessional~\cite[p.~108]{Strang09}, though I feel that
may be overstating it. Worse than useless, that dot distracts from what really
is happening. We have been well trained to understand how matrices multiply. Let
us simply write the dot product\index{Dot Product|see{Inner Product}} (or inner
product\index{Inner Product}) for two column vectors
$\mathbfit{a}\tr\mathbfit{b} = \sum_j a_j b_j$. It is unsurprising then that the
outer product (or rank one product, or \textit{tensor} product) is
$\mathbfit{a}\mathbfit{b}\tr$.

The matrix multiplication really lets us connect with the underlying linear
algebra. It does work for us, even when we extend linear algebra to functions
(that is, functional analysis). Let us think about two functions of a real
variable, $f(t)$ and $g(t)$. Let us also discretise by evaluating on a mesh of
$t$ that is sampled with interval $T$. More concisely, $t = kT$ with
$k=0,1,2,\ldots$ . We can then think of $f$ as a column of values $f_k = f(kT)$.
The inner product notation now suggests a convenient form for the inner product
of two functions:
\[
  f\tr g = \sum_j f_j\, g_j.
\]
Now, work your way back to the continuous case by allowing $T\rightarrow 0$, and
the above sum condenses to the integral:
\[
  \lim_{T\rightarrow 0}\, \sum_j f_j\, g_j = \int_{-\infty}^\infty f(x)\, g(x)\,\dif x = \bigl<f,\, g\bigr>.
\]
It is short work from here to derive the Fourier or Laplace transformations, and
we were well set up for it using the right notation. (If you are unsure about
getting Fourier or Laplace out of this, consider taking the inner product of a
function of interest with each member of a set of orthogonal basis functions,
such as the complex sinusoids. You are decomposing the function vector in a
Hilbert space of functions just as one would find the $x$-, $y$- and
$z$-components of a vector in Euclidean 3-space.)

I digress for a moment to appreciate that combining the above definition of the
inner product with the transpose identity \eqref{eq:transpose} can be used to
derive integration by parts:
\begin{align*}
  (\mathbfit{A\,x})\tr\,\mathbfit{y} &= \mathbfit{x}\tr\,(\mathbfit{A\,y})\\
  \int_{-\infty}^\infty \diff{f}{t}\, g(t)\,\dif t &= \int_{-\infty}^\infty f(t)\left(-\diff{g}{t}\right)\,\dif t.
\end{align*}

The reader can expect two notations for the differential operation. I could
hardly be accused of rebellion for adopting the common notation of Leibniz,
\begin{equation*}
  \diff{y}{x} = \diff{}{x}\,f,
\end{equation*}
for the derivative of $y$ with respect to $x$. I will also exercise the more
compact notation for the same thing:
\begin{equation*}
  \diff{y}{x} = \pdop{x}{y}.
\end{equation*}
The former notation has the disadvantage of becoming optically obscure if $x$
were to have super/sub-scripts, so in such cases I will fall back to
Leibniz.\marginnote{The notation with the \emph{del}, $\partial$, is popular in
field theory (and related areas of physics) and I believe is based on the
notation of Oliver Heaviside which is the same except using a D in place of the
del.}

The set of real numbers is denoted by \reals\index{Real@\reals\ (real numbers)}.
The symbol \posReals\ is a common notation for the positive real numbers. It is
also a common short hand for the additive group of real numbers, $(\reals, +)$.
I will surely reference the positive real numbers much more frequently, so the
compact notation \posReals\ is reserved for that. Similarly, \posInts\index{Z
Integers@\ints, (Integers)}\index{Integers!set of|see{\ints}}\ will represent
positive integers, which some may refer to as natural numbers. Still others
would not recognize these as natural numbers as there is no general agreement on
whether natural numbers should be the positive integers, or the non-negative
integers. To avoid confusion, I notate the non-negative integers as
$\nonNegInts = \{\,0\,\} \cup \posInts$, and avoid the phrase \emph{natural
numbers} everywhere but this paragraph.

The Kronecker-delta, $\kron^n_m=\kron_{nm}$\index{Kronecker-delta, $\kron_n^m$},
(which evaluates to $1$ if and only if $n=m$ and $0$ otherwise), is upright. So
too, is the Dirac-delta\index{Dirac-delta, $\dirac(x)$}, $\dirac(x)$, which is
zero everywhere except at $x=0$ with $\int_{-\infty}^\infty \dirac(x)\,\dif x =
1$.

In this document, there will be references to discretely sampled trajectories
through (flat) spaces of arbitrary dimension. For example, a trajectory through
a 7-dimensional Euclidean space. These sampled trajectories will be collected as
a time series into a structure that looks a lot like a matrix: a row of
column-vectors forming a rank-2 array. These will typically be notated with a
lowercase bold italicized Latin symbol, such as \q, or \cu---breaking the
convention of uppercase for matrices. This departure from the convention
indicates that these structures are most usefully thought of as a time series of
column-vectors. Take such an entity $\bia \in \reals^{n\times N}$. The $k$-th
column-vector in the series will be notated as $\bia^k$ with $0 \le k \le N-1$.
Since these are vector quantities, there should be no confusion with
exponentiation. The superscript notation frees the subscript position for
notational embellishments. This sort of notation is not strange, it is simply
borrowed from tensor notation. In tensor notation, there is a difference between
super-scripts (which indicate covariant components) and subscripts (which
indicate contravariant components). However, we will be working in flat spaces
where the two are equivalent. However, I would rather not even confuse the issue
with such concerns. Simply keep in mind that superscripts on vectors indicate a
column-number in a series of column vectors.

Parentheses, brackets and braces will be used with consistency. Parentheses,
$(\,\cdot\,)$, will be used as traditional delimiters.

Braces, $\{\,\cdot\,\}$, identify sets. For example, $\bigl\{a_1,\,
a_2,\ldots,a_N \bigr\}$. I also make use of the ranged brace notation, which
expresses that last set as $\bigl\{a_k\bigr\}_{k=1}^N$.

Brackets, $[\,\cdot\,]$, will be used to enclose composite structures such as
vectors and matrices, and they will also be used as outer delimiters for tall
set operators, such as $\sum$, $\prod$, $\int$, and $\bigcup$. For example, the
definition of the Euclidean metric:
\begin{equation}\label{eq:Euclidean-metric}
  d(\bia,\,\bib) = \norm{\bia - \bib} = \Bigl(\sum_i\,\bigl[a_i-b_i\bigr]^2\Bigr)^{1/2}
\end{equation}

Of course, that definition for the Euclidean metric is based on the Euclidean,
or $l^2$-norm:
\begin{equation*}
  \norm{\bix} = \Bigl(\sum_i x_i^2\Bigr)^{1/2}.
\end{equation*}

On a normed space $\Omega$, the norm is written $\norm{\cdot}_\Omega$.
Similarly, the metric will be written $d_\Omega(\cdot,\, \cdot)$. Of course,
that norm and metric may or may not be Euclidean, and the presence of that
subscript should incite to question.

The class of $n$-times continuously differentiable functions is denoted
$\continuouslyDifferentiable[n]$.%
\index{cfunctions@\continuouslyDifferentiable[n]-functions} The derivatives
should be bounded so that the supnorm,
%
\begin{equation*}
  \norm{f}_{\continuouslyDifferentiable[k](\Omega)} = \sum_{n=0}^{k}\sup_{x\in\Omega}\;\Biggl\| \diff[n]{f}{x} \Biggr\|,
\end{equation*}
%
can complete the Banach space, which has important analytical consequences.

\section[Conventions \emph{for} Matrix Calculus]%
{conventions \emph{for} matrix calculus}%
\label{sec:matrix-calc}


Two conventions exist for organising the calculus on matrix and vector
quantities. Namely, these are the \emph{numerator-} and
\emph{denominator-}layouts\index{Numerator-layout}\index{Denominator-layout}.
The two layouts are related (mostly) by transposition. For this document I adopt
the numerator layout, which may be exemplified in the following five cases.
\begin{enumerate}
  \item For a scalar valued function of a vector, $y : \reals^n\rightarrow \reals$, the gradient is a row-vector: \begin{equation*}
    \pdiff{y}{\bix} = \begin{bmatrix}
     \pdiff{y}{x_1} & \pdiff{y}{x_2} & \cdots & \pdiff{y}{x_n}\\
     \end{bmatrix}.
    \end{equation*}
  \item For a vector valued function of a scalar, $\biy : \reals\rightarrow\reals^n$, the derivative is a column-vector: \begin{equation*}
    \pdiff{\biy}{x} = \begin{bmatrix}
     \ipdiff{y_1}{x} \\ \ipdiff{y_2}{x} \\ \vdots \\ \ipdiff{y_n}{x}
     \end{bmatrix}.
    \end{equation*}
  \item For a vector valued function of a vector, $\biy: \reals^n\rightarrow\reals^m$, the derivative is the matrix:  \begin{equation*}\renewcommand{\arraystretch}{1.8}
    \pdiff{\biy}{\bix} = \begin{bmatrix}
      \pdiff{y_1}{x_1} & \pdiff{y_1}{x_2} & \cdots & \pdiff{y_1}{x_n}\\
      \pdiff{y_2}{x_1} & \pdiff{y_2}{x_2} & \cdots & \pdiff{y_2}{x_n}\\
          \vdots       &      \vdots      & \ddots &       \vdots     \\
      \pdiff{y_m}{x_1} & \pdiff{y_m}{x_2} & \cdots & \pdiff{y_m}{x_n}\\
    \end{bmatrix}.
  \end{equation*}
  \item For a scalar valued function of a matrix, $y : \reals^{m\times n}\rightarrow \reals$, the derivative is a matrix: \begin{equation*}\renewcommand{\arraystretch}{1.8}
    \pdiff{y}{\biX} = \begin{bmatrix}
      \pdiff{y}{X_{11}} & \pdiff{y}{X_{21}} & \cdots & \pdiff{y}{X_{m1}}\\
      \pdiff{y}{X_{12}} & \pdiff{y}{X_{22}} & \cdots & \pdiff{y}{X_{m2}}\\
           \vdots       &      \vdots       & \ddots &      \vdots      \\
      \pdiff{y}{X_{1n}} & \pdiff{y}{X_{2n}} & \cdots & \pdiff{y}{X_{mn}}\\
    \end{bmatrix}.
  \end{equation*}
  \item And finally, for a matrix valued function of a scalar, $\biY : \reals\rightarrow \reals^{m\times n}$, the derivative (which is only defined in numerator-layout) is a matrix: \begin{equation*}\renewcommand{\arraystretch}{1.8}
    \pdiff{\biY}{x} = \begin{bmatrix}
      \pdiff{Y_{11}}{x} & \pdiff{Y_{12}}{x} & \cdots & \pdiff{Y_{1n}}{x}\\
      \pdiff{Y_{21}}{x} & \pdiff{Y_{22}}{x} & \cdots & \pdiff{Y_{2n}}{x}\\
          \vdots       &       \vdots       & \ddots &      \vdots     \\
      \pdiff{Y_{m1}}{x} & \pdiff{Y_{m2}}{x} & \cdots & \pdiff{Y_{mn}}{x}\\
    \end{bmatrix}.
  \end{equation*}
\end{enumerate}
A quantity simply stated to be a \emph{vector} should be considered a
column-vector unless otherwise indicated.

  \clearpage
  \printacronyms[name={List of Acronyms, Initialisms and Selected Jargon}, heading=chapter]
  \clearpage
  \chapter{Our Robotics Publications}%
  \label{chap:Pubs-Robotics}
    This appendix indexes copies of the following two articles:
    \begin{enumerate}
      \item \fullcite{Teatro2013}
      \item \fullcite{Teatro2014}
    \end{enumerate}
    If you are reading a hard copy of this document, an individually stapled copies of these documents are placed hereafter. Readers of an electronic copy of this document may find them in in PDF form in the same directory.
  \clearpage
    \chapter{Our Reactor Control Publication}%
    \label{chap:Pubs-Nuclear}
    This appendix indexes a copy of the following article:
    \begin{itemize}
      \item\fullcite{Teatro2015}
    \end{itemize}
    If you are reading a hard copy of this document, an individually stapled copy of this document is placed hereafter. Readers of an electronic copy of this document may find it in in PDF form in the same directory.
\clearpage
    \chapter{Our Software Design Publication}%
    \label{chap:Pubs-Software}
    This appendix indexes a copy of the following article:
    \begin{itemize}
      \item\fullcite{Teatro2016}
    \end{itemize}
    If you are reading a hard copy of this document, an individually stapled copy of this document is placed hereafter. Readers of an electronic copy of this document may find it in in PDF form in the same directory.
\clearpage
\backmatter%
  {
    \RaggedRight%
    \renewcommand\bibfont{\footnotesize}
    \printbibliography%
  }
  \printindex

\end{document}
% EOF
