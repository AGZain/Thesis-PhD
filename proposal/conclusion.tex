%!TEX root = Proposal-PhD.tex
%
\chapter{Conclusion}%
\label{chap:conclusion}

\textsc{The complexity of the control tasks} intrinsic to many of the
technologies on the horizon is producing proportionally complex software. At the
industry's leading edge management of software complexity is an ongoing issue.
New software development techniques and architectural styles are developed to
improve software quality as a means of controlling complexity.

The preceding document has outlined a course of research expanding modern
control techniques \emph{and} software designs that enable more rapid progress
is \ac{nmpc} control systems research. I desire to take lessons from the leading
edge of the software industry and from computer science to discover techniques
and best practices for designing software for \ac{nmpc} that is scalable,
distributable and parallelisable.

For the proposal, I outline three main areas of contribution:
\begin{enumerate}[label=(C\arabic*)]
  %
  \item the \acl{sdd}.
  %
  \item the software capital, and
  %
  \item novel theoretical research in the field of optimal control
  systems.
  %
\end{enumerate}

In a test-driven development process, I wish to develop a modern framework using
functional, reactive programming techniques and the modern \ac{cpp17} language.
The product of that endeavour covers the first two contributions.

The proposed framework is to be the basis for several research projects,
\textit{in potentiâ}. For example, one such project investigates the use of
artificial neural networks to accelerate the optimisation stage of the
algorithm, based on a sampling of the potential field which simulates obstacles
in state space. That project, and the others hold novelty in the field of
control systems research and represent progress in \ref{contrib:theory}. They
additionally demonstrate the usefulness of the framework which underlies them.
