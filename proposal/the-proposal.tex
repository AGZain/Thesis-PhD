%!TEX root = Proposal-PhD.tex
%
\chapter{The Proposal}%
\label{chap:proposal}


In the introductory chapter, \Sref{chap:introduction}, I enumerated three
primary contributions of my thesis:
%
\MarginDefinition{A \textbf{\textit{\acl{sdd}}} is, as outlined in \ac{ieee1016}
is a description of a software product that \textit{can be produced to capture
one or more levels of concern with respect to its design subject. These levels
are usually determined by the design methods in use or the life cycle context;
they have names such as “architectural design,” “logical design,” or “physical
design.} I will use the term to describe, mainly, the concerns of what is done
within the software components and their interrelationships, as opposed to
\emph{how} anything is done.}
%
\begin{enumerate}[label=(C\arabic*)]
  %
  \item the \textbf{\textit{\acl{sdd}}}. \label{contrib:design}
  %
  \item the \emph{software capital}, and \label{contrib:capital}
  %
  \item novel theoretical research in the field of optimal control
  systems.\label{contrib:theory}
  %
\end{enumerate}
In this chapter, I shall attempt to itemise and prioritise some milestones,
principles and guidelines for the proposed research. It is important to keep in
mind that, at any given stage of research, the forward direction will be
determined by intermediate findings. At this point, it is impossible to predict
those directions, and making plans that are too specific may, at best, be a
misuse of attention and energy. So, I will focus on describing, in general
terms, parameters and values that will guide and inform my decisions along the
way.

The \ac{oo}-paradigm organises code to encapsulate the moving parts. Functional
code reduces the number of moving parts. When implemented puritanically, neither
of these paradigms is a panacea. The best way to think about predictive control
systems is certainly found in a balance between these two philosophies. I wish
to explore that balance and describe some rules and patterns which merge the
best from each.
