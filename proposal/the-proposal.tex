%!TEX root = Proposal-PhD.tex
%
\chapter{The Proposal}%
\label{chap:proposal}


In the introductory chapter, \Sref{chap:introduction}, I enumerated three
primary contributions of my thesis:
%
\MarginDefinition{A \textbf{\textit{\acl{sdd}}} is, as outlined in \ac{ieee1016}
is a description of a software product that \textit{can be produced to capture
one or more levels of concern with respect to its design subject. These levels
are usually determined by the design methods in use or the life cycle context;
they have names such as “architectural design,” “logical design,” or “physical
design.} I will use the term to describe, mainly, the concerns of what is done
within the software components and their interrelationships, as opposed to
\emph{how} anything is done.}
%
\begin{enumerate}[label=(C\arabic*)]
  %
  \item the \textbf{\textit{\acl{sdd}}}. \label{contrib:design}
  %
  \item the \emph{software capital}, and \label{contrib:capital}
  %
  \item novel theoretical research in the field of optimal control
  systems.\label{contrib:theory}
  %
\end{enumerate}
In this chapter, I shall attempt to itemise and prioritise some milestones,
principles and guidelines for the proposed research. It is important to keep in
mind that, at any given stage of research, the forward direction will be
determined by intermediate findings. At this point, it is impossible to predict
those directions, and making plans that are too specific may, at best, be a
misuse of attention and energy. So, I will focus on describing, in general
terms, parameters and values that will guide and inform my decisions along the
way.



\section{Research Objectives for the Software Design}


\textsc{Item} \ref{contrib:design} in my list of contributions is the
architectural description of the software framework, the \ac{sdd}. It can, in
places, be difficult to decouple \ref{contrib:design} from
\ref{contrib:capital}, which embodies the actual code. The architecture consists
of descriptions that comport with \acl{ieee1016} (and possibly
\acl{ieee420101}), and can be implemented in any \emph{Turing complete}
language. But the language can have a large impact on the form of patterns and
contracts.

For example, in a strongly typed language like C++ or Java, interface
specifications are an important component of polymorphism. In \emph{duck typed}
languages like Python, interfaces are meaningless, and inheritance is only used
to share implementation between classes.

The \ac{sdd} will outline the software components, their contractual obligation
to the user and their interrelationships. However, the design of any given
component should be flexible and open to easy modification and adaptation. So, I
expect that some of the strongest contributions will be in the form of patterns
and reusable design solutions for common problems in control systems software.


\section{Research Objectives for the Software Capital}%
\label{sec:ObjectivesForSoftwareCapital}


\begin{flushright}
\begin{minipage}{0.85\textwidth}
  \footnotesize
  {\color{git-ltgray}
  \begin{verbatim}
    Python 3.5.2
    Type "help", "copyright", "credits" or "license" for more information.
  \end{verbatim}
  }\null\vskip-3\baselineskip
  {\ttfamily
  \noindent {>}{>}{>} \textcolor{git-red}{import} this
  }
  \begin{verbatim}
    The Zen of Python, by Tim Peters

    Beautiful is better than ugly.
    Explicit is better than implicit.
    Simple is better than complex.
    Complex is better than complicated.
    Flat is better than nested.
    Sparse is better than dense.
    Readability counts.
    Special cases aren't special enough to break the rules.
    Although practicality beats purity.
    Errors should never pass silently.
    Unless explicitly silenced.
    In the face of ambiguity, refuse the temptation to guess.
    There should be one-- and preferably only one --obvious way to do it.
    Although that way may not be obvious at first unless you're Dutch.
    Now is better than never.
    Although never is often better than *right* now.
    If the implementation is hard to explain, it's a bad idea.
    If the implementation is easy to explain, it may be a good idea.
    Namespaces are one honking great idea -- let's do more of those!
  \end{verbatim}
\end{minipage}\\
\EpiName{Tim Peters}
\EpiSource{The Zen of Python}
\end{flushright}

\noindent\textsc{I open this section with a poem} by the great Tim Peters, the
software engineer and computer scientist of \texttt{timsort}\SideNote{timsort is
the algorithm used in the standard sorting commands in Python, Java SE 7, on
Android and in GNU Octave.} fame. (And if you did not know about it, yes, you
can import the |this| module in python, and it prints out Peters' poem.) The
poem, called \emph{The Zen of Python}, expresses some values of software
craftsmanship that comport well with the goals of the software component of the
proposed research. Those values largely emphasise readability and clarity over
clever shortcuts.

As I learn and understand more about functional programming, I realise that
there may be value in code that is \emph{hard to explain}, seemingly in
contrition to Peters' dictum. But the difficulty in explaining functional
designs is not from needless complexity. It comes from a cultural barrier. As
students, we simply are not taught to think and understand functional patterns
in programming and one incurs that overhead, at least once, when seeing it for
the first time. Once someone has overcome the learning curve, functional designs
can be very clean and understandable when compared to equivalent procedural
code. (For a practical but anecdotal investigation, see the \ac{darpa} sponsored
study by \citeauthor{Haskell-vs-ada} in
\citetitle{Haskell-vs-ada}~\cite{Haskell-vs-ada}.)

The \ac{oo}-paradigm organises code to encapsulate the moving parts. Functional
code reduces the number of moving parts. When implemented puritanically, neither
of these paradigms is a panacea. The best way to think about predictive control
systems is certainly found in a balance between these two philosophies. I wish
to explore that balance and describe some rules and patterns which merge the
best from each.

Regardless of the principles I uncover in the course of the proposed research,
the following list of design goals will remain at high priority:
%
\begin{description}
  %
  \item[Compliant.] It should comply strictly with \acl{cpp17} and the C++ Core
  Guidelines~\cite{CppCoreGuidelines}\SideNote{From the GitHub project
  description: \emph{The C++ Core Guidelines are a set of tried-and-true
  guidelines, rules, and best practices about coding in C++}}. This will include
  use of the \ac{gsl} wherever appropriate.
  %
  \item[Practical.] Code which is high quality and production ready. It should
  be efficient enough for real-world use. It should be straightforward to use,
  read and maintain. It will conform to a formatting style that optimises for
  readability. My metric for this item will be this question: \emph{can a clever
  and well motivated graduate student get a project up-and-running in less than
  a week?}
  %
  \item[Unit Tested.] The entire codebase, including examples, should be covered
  by unit tests. If a user begins a new design based on the example code, they
  can adapt the tests as needed. Of course, \SI{100}{\percent} test coverage is
  a rarely achieved goal, but it is always a good goal to have.
  %
  \item[Lightweight.] The framework itself should be lean enough that a control
  system designer can, if they are able to cope with the numerics, deploy the
  framework on modest hardware, such as Arduino, Raspberry PI, Intel Edison, or
  TI Beagle Bones. Emphasis will be placed on abstraction that is zero-overhead
  or low-overhead, but never at the cost of clarity.
  %
  \item[Portable.] It should compile with the latest \ac{gcc} C++ compiler and
  Clang++ compilers. The code will compile without warnings using the
  |--pedantic| compiler flag.
  %
  \item[Pattern-oriented.] Emphasis is placed on using patterns in order to make
  the code more easily understood. The framework needs to be readily extensible
  by control systems researchers. The use of patterns makes the code more
  expressive, and suggests good techniques and practices to those extending the
  code.
  %
  \item[Minimal mutable state.] In the context of software, a mutable state is a
  feature of routines that store objects and change values during runtime.
  Whenever you write code with an assignment outside of initialisation, you are
  modifying state. I am relatively certain the product of my research will not
  be without mutable state. But there are good reasons to avoid needless use of
  state.

  Mutable state introduces complexity and opportunities for bugs, especially in
  parallel applications. Control systems researchers who are extending the code
  should not have to worry about subtleties of the framework obstructing their
  research. Where mutable state is sufficiently sensible, I will attempt to
  abstract it away from the \ac{api}, so it does not complicate parallelism or
  distribution of the computational tasks.

  Truly stateless programs are a feature of \emph{pure functional programming}.
  However, it is not practical as a public solution, and a major part of the
  proposed research is discovering a balance that is practical within the
  context of modern C++ and the control systems community.
  %
\end{description}

The code will be documented, and a suite of examples will serve as illustrative
documentation, templates or starting points for users. Those examples will
include common problems from optimal control textbooks. Some of the selected
textbook questions include:
%
\begin{itemize}
  \item \ac{pid} control of a damped, driven harmonic oscillator
  \item \ac{nmpc} control of a train-like vehicle with throttle and break control.~\cite[p.~5]{Kirk2004}
  \item \ac{nmpc} control of artificial satellite altitude adjustment, as in~\cite[p.~1]{Vinter2010}
\end{itemize}



\section{Research Objectives in Control Systems Theory}%
\label{sec:ObjectivesForControlSystemsTheory}


\textsc{The following subsections} each describe (briefly) an avenue for
research in the theory of optimal control and \ac{nmpc}, with a view
toward~\ref{contrib:theory}. However, many of these projects also provide sample
code and documentation for contributions~\ref{contrib:design}
and~\ref{contrib:capital}.


\subsection{Analysis of Hard and Soft Constraints}

While designing the controller for virtualME\SideNote{For details on the
virtualME controller, see \Aref{chap:Pubs-Robotics}.} I created a useful
software design for handling artificial potential fields. The software still
needs refinement and extension before it is complete.

One of my interest lies in analysis and design of artificial potential fields,
and the possibility of switching constraints between hard and soft flavours
during runtime. Since soft constraints are computationally lighter, this can be
a route to increasing efficiency and reliability.


\subsection{Robotic Reflexes}

With the \ac{nmpc} algorithm in dynamic state spaces, or with dynamically
configuration artificial potential fields, there is always a concern that the
optimisation step may take too long, when the system must react very quickly to
avoid disaster. Search based optimisation strategies are faster when the
starting \emph{guess} is close to the optimum. The techniques for providing
better guesses is called \emph{hot-starting}.

I would like to try using a neural network to give a robotic system a behaviour
which resembles our human reflexes. The artificial potential field around the
robot can be sampled on a mesh. That mesh can be used as in input to the neural
network, which can either hot-start the optimisation step, or even circumvent
the optimisation step providing a suitable approximation to the \ac{nmpc}
solution.


\subsection{Comparison Among Gradient Based Search Strategies}

When I was struggling with the original \ac{c99} controller, there was a study I
was particularly interested in performing. The framework I am designing would
make it nearly trivial.

In this study, I would like to extend my already existing gradient based
controller (see \Aref{chap:Pubs-Robotics}) with a new optimiser. I would like to
build an \emph{adaptor}-class\SideNote{See the \emph{adaptor pattern} in the
\ac{gof} book~\cite{GOF}.} for the \ac{gnu-sl} optimisation routines. It will
then be trivial to switch between them.

It would be interesting to explore which routines may be most effective when
paired with certain types of artificial potential field. This relates to the
analysis of hard and soft constraints, described in the previous subsection.


\subsection{Comparison Between Gradient and \acs*{nlp} Optimisation}

The choice between gradient based and \ac{nlp} based algorithms is one faced by
control designers each time they implement an \ac{nmpc} controller. Despite that
fact, there seems to be a lack of literature comparing the two in a rigorous
way.

It is a near certainty that one technique will have advantages over the other in
certain conditions. One of my research curiosities is in the possibility of
switching between the two representations of the problem, leveraging the
advantages of each.


\subsection{Dynamic model adaptation}

Functional software design makes it extremely natural to compose functions. This
means that optimisation techniques such as \emph{genetic algorithms} are
relatively straightforward to implement.

I would like to design a supervisory algorithm which evolves the predictive
model, to learn, improve and adapt the model under dynamic conditions. This
research parallels the work done by \citeauthor{Bongard2006} in
\cite{Bongard2006}.



\section{Sequence of Research Milestones}


%
\begin{enumerate}
  %
  \item Once I have identified relevant patterns, I can continue designing the
  final architecture. The patterns will provide a layer of communication through
  nomenclature, while the more ridged components are expressed through
  contracts.
  %
  \item In the next phase, common textbook example questions will be implemented
  as examples for use. Since the solutions to these problems are well known and
  documented, they are good tests. And, their implementation as example code is
  a useful form of documentation that allows users to focus on implementation
  without the additional complexity of an unfamiliar problem.

  Some of the selected textbook questions include:
  %
  \begin{itemize}
    %
    \item \ac{pid} control of a damped, driven harmonic oscillator
    %
    \item \ac{nmpc} control of a train-like vehicle with throttle and break
    control.~\cite[p.~5]{Kirk2004}
    %
    \item \ac{nmpc} control of artificial satellite altitude adjustment, as
    in~\cite[p.~1]{Vinter2010}
    %
  \end{itemize}
  %
  \item At this stage, I should reasses the research avenues
  for~\ref{contrib:theory}, (as described
  in~\Sref{sec:ObjectivesForControlSystemsTheory}), and determine an appropriate
  order for implementing those.
  %
  \item In the final stage of development, I want to re-implement the virtualME
  controller and augment it with an algorithm to learn and improve the system
  model, as described in~\Sref{sec:ObjectivesForControlSystemsTheory}.
  %
\end{enumerate}
%



\section{Approach and Methodology}


I am developing software to further my own research goals, as
per~\Sref{sec:ObjectivesForControlSystemsTheory}. I have also laid out in
several simple example problems as part of the documentation and testing. I have
outlined comprehensive unit-test coverage as a goal in~
\Sref{sec:ObjectivesForSoftwareCapital}.

This is an ideal situation for a tight cycle of development driven by test-first
principles, such as the \acf{tdd} model of \citeauthor{Beck2003}, as expounded
in his \citeyear{Beck2003} book \citetitle{Beck2003}~\cite{Beck2003}. The
\ac{tdd} method has proven very effective, and is often considered to be a
component of the \emph{Agile software development} and \emph{Extreme
programming} development methods, and others.

The benefits of \ac{tdd} include
%
\begin{itemize}
  %
  \item allowing us to catch bugs before the cause problems,
  %
  \item future-proofs against new bugs in an evolving codebase
  %
  \item projects an image of dependability and a mechanism for proving it
  %
  \item facilitates safe re-factoring when users adapt it to their needs.
  %
\end{itemize}

There are many popular books prescribing \ac{tdd}, such as
\citeauthor{Martin2009}'s \citetitle{Martin2009}~\cite{Martin2009}, or the
aforementioned \citeauthor{Beck2003}~\cite{Beck2003} or his later
\citetitle{Beck2007}~\cite{Beck2007}. But most importantly, there are also
academic works advocating for the method. For example, the HydroShare case study
in \citetitle{Carver2016} by \citeauthor{Carver2016}~\cite{Carver2016}, or the
book \citetitle{Kandt2006} by \citeauthor{Kandt2006}~\cite{Kandt2006}.
\citeauthor{Kandt2006}'s book expounds the use of \emph{extreme programming}, of
which \ac{tdd} is an integral part.

Most pertinently, a \anno{2005} study commissioned by Canada's National Research
Council found that \ac{tdd} can lead to higher productivity~\cite{Erdogmus2005}.
In particular, they conclude
%
\begin{quote}
  %
  Our main result is that Test-First programmers write more tests per unit of
  programming effort. In turn, a higher number of programmer tests lead to
  proportionally higher levels of productivity. Thus, through a chain effect,
  Test- First appears to improve productivity. We believe that advancing
  development one test at a time and writing tests before implementation
  encourage better decomposition, improves understanding of the underlying
  requirements, and reduces the scope of the tasks to be performed.
  %
\end{quote}

I also want to investigate the fusion of \acf{dbc}~\cite{Meyer1992} with
\ac{tdd} in C++. The \ac{cpp17} standard has no native support for contract
specification, though there are some implementations where some have adapted
|assert| for those purposes. However, I think \ac{tdd} provides a mechanism to
define and test contractually defined behaviour. Each \emph{precondition},
\emph{postcondition} and \emph{invariant} can be the subject of a unit-test.
